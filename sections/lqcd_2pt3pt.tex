In order to determine the mass of the nucleon (or any hadron), we rely on the construction of two-point correlation functions that are constructed in a mixed time-momentum representation, most commonly utilizing spatially local creation operators (sources) and momentum space annihilation operators (sinks).
The non-perturbative nature of QCD means we do not know how to construct the nucleon wave function, and so we utilize \textit{interpolating operators} which have the quantum numbers of the state we are interested in.
These creation and annihilation operators will couple to all eigenstates of QCD with the same quantum numbers giving rise to a two-point function with a spectral decomposition
\begin{align}\label{eq:2pt}
    C(t,\mathbf{p}) &= \sum_{\mathbf{x}} e^{-i \mathbf{p\dotp x}}
        \langle \O| O(t,\mathbf{x}) O^\dagger(0,\mathbf{0}) | \O \rangle
%\nonumber\\&=
    =
    \sum_{n=0}^\infty z_n(\mathbf{p}) z_n^\dagger(\mathbf{p}) e^{-E_n(\mathbf{p})t}\, .
\end{align}
In this expression, $|\O\rangle$ is the vacuum state,
$z_n(\mathbf{p}) = \sum_{\mathbf{x}}e^{-i\mathbf{p\dotp x}} \langle \O|O(0,\mathbf{x})|n\rangle$
and $z_n^\dagger(\mathbf{p}) = \langle n(\mathbf{p})|O^\dagger(0,\mathbf{0})|\O\rangle$.
To go from the first equality to the second, we have inserted a complete set of states, $1=\sum_n |n\rangle\langle n|$ and we have used the time-evolution operator to shift the annihilation operator to $t=0$ and expose the explicit time dependence.%
%-------------------------------------------------------------------------------
\begin{marginnote}
\entry{$O(t,\mathbf{x}) = e^{\hat{H}t} O(0,\mathbf{x}) e^{-\hat{H}t}$}{The Hamiltonian, $\hat{H}$, is used to time evolve the operator}
\end{marginnote}%
%-------------------------------------------------------------------------------
Momentum conservation selects only the state $n(\mathbf{p})$ in the sum over all states enabling the use of this mixed spatial creation operator and momentum annihilation operator.
As we will discuss in more detail below, it is more desirable to instead use momentum space creation operators.
They are not commonly used as they are significantly more expensive numerically to generate.

For large Euclidean time, the correlation function will be dominated by the ground state as the excited states will be exponentially suppressed by the energy gap
\begin{equation}
    C(t) = z_0 z_0^\dagger e^{-E_0 t}\left[
        1 + r_1 r^\dagger_1 e^{-\Delta_{1,0}t} + \cdots \right]\, .
\end{equation}%
\begin{marginnote}
\entry{$\D_{m,n}= E_m - E_n$}{}
\entry{$r_n = z_n / z_0$}{}
\end{marginnote}%
It is useful to construct an \textit{effective mass} to visualize at which time $t$, the ground state begins to saturate the correlation function
\begin{align}
m_{\rm eff}(t) &= \ln \left( \frac{C(t)}{C(t+1)} \right)
%\nonumber\\&=
    =
    E_0 + \ln\left( 1 + \sum_{n=1} r_n r^\dagger_n e^{-\Delta_{n,0}t}\right)\, .
\end{align}


A well known challenge which complicates the analysis is that for nucleon two-point functions, the S/N ratio degrades exponentially at large Euclidean time~\cite{Lepage:1989hd}
\begin{equation}
\lim_{{\rm large}\ t} {\rm S/N}%\frac{\mathrm{Signal}}{\mathrm{Noise}}
    \propto \sqrt{N_{\mathrm{sample}}} e^{(m_{\mathrm{N}} - \frac{3}{2}m_\pi)t}\, ,
\end{equation}
which adds an extra source of excited state systematic uncertainty that must be quantified:
the region in time when the ground state begins to saturate the correlation functions, at $t\approx 1$~fm, the noise is becoming significant, making the correlation functions in this region susceptible to correlated fluctuations which can bias a simplistic single-state analysis.
Further, as the pion mass is reduced towards its physical value, the excited state energy gap also shrinks, as the lowest lying excited state is typically a nucleon-pion in a relative $P$-wave.  At the same time, the energy scale which governs the exponential degradation of the signal also grows.  The former issue means calculations must be performed at larger Euclidean time to suppress the smaller gapped excited states and the latter issue means we need exponentially more statistics to obtain a fixed relative uncertainty at a given time.
In order to boost statistics, given the numerical cost of generating more configurations, time and spatial translation invariance are used to generate sources at several choices of $t_0$ and $\mathbf{x}_0$, which are then translated back to the origin and averaged together.  For the nucleon, it was observed that hundreds of sources on each configuration, using an anisotropic clover-Wilson ensemble~\cite{HadronSpectrum:2008xlg}, still showed approximate $\sqrt{N_{\rm sample}}$ improvement of the stochastic uncertainty as a function of the number of sources per configuration~\cite{Beane:2009kya}.



The most common method of constructing three-point correlation functions follows a similar strategy, beginning with spatially local sources.
A nucleon three-point function with current $j_\G$ is constructed with interpolating operators $N(\tsep,\mathbf{x})$ and $N^\dagger(0,\mathbf{0})$,
\begin{align}
C_\G(\tsep,\t) &= \sum_{\mathbf{x,y}}e^{-i\mathbf{p\dotp x} +i\mathbf{q\dotp y}}
    \langle\O|N(\tsep,\mathbf{x}) j_\G(\t,\mathbf{y}) N^\dagger(0,\mathbf{0}) |\O\rangle
\nonumber\\&=
    \sum_{n,m} z_n(\mathbf{p})z_m^\dagger(\mathbf{p-q})e^{-E_n(\tsep-\t)}e^{-E_m \t} g_{n,m}^\G\, ,
\end{align}
where $g_{n,m}^\G$ are the matrix elements of interest, and in principle, all other quantities can be determined from the two-point function, a point we will return to.
Often, the sink is projected to zero momentum, $\mathbf{p}_n=0$, for which momentum conservation gives the momentum of the incoming state to be $\mathbf{p}_m = -\mathbf{q}$.%
%-------------------------------------------------------------------------------
\begin{marginnote}
\entry{$j_\G = \bar{q}\, \G\, q$}{quark bilinear currents of Dirac structure $\G$ and unspecified flavor structure}
\entry{$g_{n,m}^\G = \langle n| j_\G |m\rangle$}{hadronic matrix elements of interest from state $m$ to $n$ with implicit momentum and energy dependence}
\entry{$\tsep$}{The time-separation between the sink and source}
\end{marginnote}%
%-------------------------------------------------------------------------------
A typical calculation is performed with a sequential propagator~\cite{Martinelli:1988rr}
whose source is constructed by taking the forward propagators from the origin and contracting all but one spin and color index at the sink time, $\tsep$.%
% FOOTNOTE ---------------------------------------------------------------------
\footnote{There are alternative methods for computing nucleon structure known as the one-end trick~\cite{Foster:1998vw,McNeile:2006bz,Alexandrou:2013xon} and a variant of the Feynman-Hellmann Theorem~\cite{CSSM:2014uyt}, but these are not in wide use.}
%-------------------------------------------------------------------------------

Each choice of $\tsep$, flavor and spin of the initial and final state, and each choice of flavor for the current requires a new sequential propagator, rendering this aspect of the computation relatively expensive.
While each sequential propagator can only be used for these specific choices, one is free to insert any quark bilinear operator for the current, including non-trivial spatial structure and momentum.
To boost statistics, in addition to taking advantage of translation invariance as with the two-point function, the \textit{coherent sink technique}~\cite{LHPC:2010jcs} is used to solve a single sequential propagator for many choices of the origin.

The most significant challenge in determining the nucleon matrix elements and subsequent form factors is dealing with the excited state contamination, an issue that is compounded by the degrading S/N.
If the nucleon two-point function is becoming saturated by the ground state at $t\approx1$~fm, the ideal three-point function would use values of $\tsep\approx2$~fm.
In practice, a few values of $\tsep$ in the range $0.8 \lesssim \tsep \lesssim 1.5$~fm are used because the S/N ratio of the three-point functions decays more rapidly than the two-point function, and an extrapolation to large $\tsep$ is used to isolate the ground state matrix elements.
However, using just one excited state in the analysis requires three values of $\tsep$ in order to have a one degree-of-freedom fit to extract the ground state matrix element.
Most results have been generated with three or fewer values of $\tsep$.
Following Ref.~\cite{Chang:2018uxx}, that utilized many values of $\tsep$ to determine $g_{\mathrm{A}}$, a few groups have begun advocating for the use of many values of $\tsep$, including small values, to improve our ability to understand and control the excited state contamination~\cite{Hasan:2019noy,Alexandrou:2019brg,He:2021yvm}.
Ref.~\cite{He:2021yvm} utilized 13 values of $\tsep$, which allows for a systematic study of the uncertainty associated with truncating early and/or late values of $\tsep$ in the analysis, as well as providing sufficient data to perform up to a complete 5-state fit.

At non-zero momentum, the excited states become more problematic.  While the energy associated with the signal scales with the momentum, the energy scale associated with the noise is independent of the momentum and thus their difference, which is the energy scale that governs the decay of the S/N, grows with increasing momentum.
For values of $Q\gtrsim 2$~GeV, the noise becomes unmanagable unless one uses a smearing profile that couples more strongly to boosted nucleons~\cite{Bali:2016lva}.
For boosted nucleons, parity is also no longer a good quantum number and so the correlation function begins to mix even and odd parity states through the matrix elements.  Such contamination can be handled through a variational method that incoporates even and odd parity nucleons~\cite{Stokes:2013fgw,Stokes:2018emx,Stokes:2019zdd}.



Recent results have uncovered some additional aspects of excited state contamination:
\begin{itemize}[leftmargin=*]
\item Two-point functions constructed as in Equation~\eqref{eq:2pt} are insufficient to reliably determine any of the excited states.  Different choices of $\tmin$ and different reasonable priors in a Bayesian analysis support excited states that differ by several sigma, also resulting in sensitivity of the ground state spectrum~\cite{Park:2021ypf,He:2021yvm} and matrix elements~\cite{Jang:2019vkm,Gupta:2021ahb}.

\item In contrast, the curvature in the three-point functions associated with the current insertion time is very constraining on the excited state spectrum, resulting in very stable and robust determination of the spectrum while varying $\tmin$, the number of excited states, the excited state model and the excited state priors~\cite{He:2021yvm}.
While it is encouraging that the spectrum and matrix elements become very stable, such an analysis offers no insight into the nature of the excited states.  For example, with values of $m_\pi L\approx4$, the $P$-wave $N(\mathbf{p})\pi(-\mathbf{p})$ excited state energy gap is essentially the same as the $N\pi\pi$ threshold state.

\item Many groups determine the spectrum and overlap factors from fits to the two-point functions, and then pass these results into the three-point function analysis without allowing the values to adjust to the global minimum.  Such a choice can either lead to an overestimate of the uncertainty of the three-point functions (if one uses the variablility of the spectrum mentioned above), or a biased extraction of the ground state matrix elements (if one uses the ``wrong'' value of the spectrum).  Given the computational setup described above, the robust choice is to perform a global analysis of the two- and three-point functions simultaneously.

\item For zero-momentum transfer, where the spectrum of the in and the out states have the same value,
the use of the summation method~\cite{Maiani:1987by} or a variant of the Feynman-Hellman method~\cite{deDivitiis:2012vs,Bouchard:2016heu} which is closely related to the summation method, it is understood that the excited states become significantly suppressed compared to the three-point function~\cite{Capitani:2012gj,He:2021yvm}.
For non-zero momentum transfer, such a technique can only be used well in the Breit-Frame where the momentum of the in and out states is equal and opposite, such that the energy spectrum is the same~\cite{Gambhir:2019pvw}.

\item The excited state contamination is particularly relevant for the PCAC relation, which we discuss in more detail in Section~\ref{sec:lqcd_pcac}.


\end{itemize}
