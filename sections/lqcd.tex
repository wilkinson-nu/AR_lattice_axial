% ------------------------------------------------------------------------------
% LQCD Intro
\subsection{Lattice QCD\label{sec:lqcd_intro}}
Lattice QCD is a non-perturbative regularization of QCD that is amenable to numerical calculations



% ------------------------------------------------------------------------------
% LQCD results
\subsection{Survey of lattice QCD results of $F_A(Q^2)$\label{sec:lqcd_results}}

Citations for $F_A(Q^2)$ references pulled from NME21, no $g_A$-only references:
\begin{description}
\item[NME 21]~\cite{Park:2021ypf}
\item[RQCD 20]~\cite{Bali:2018qus,RQCD:2019jai} %% 63
\item[ETMC 20]~\cite{Alexandrou:2018sjm,Alexandrou:2019brg,Alexandrou:2020okk} %% 54-56
\item[PACS 18 (erratum)]~\cite{Ishikawa:2018rew,Shintani:2018ozy} %% 60-61 (62 proceedings)
\item[PNDME 17]~\cite{Gupta:2017dwj,Gupta:2018qil,Jang:2019vkm,Jang:2019jkn} %% 6-9
\item[CLS 17]~\cite{Hasan:2017wwt,Hasan:2019noy} %% LHPC? 66-67
\end{description}


% ------------------------------------------------------------------------------
% z-expansion
\subsection{Combining the $z$-expansion with the continuum and chiral extrapolations\label{sec:z_continuum}}

The $z$ expansion is a conformal mapping from the four-momentum transfer squared $Q^2$
 to a small expansion parameter $z$, parameterized by the relation
\begin{align}
 z(t=-Q^2;t_0,t_c) = \frac{\sqrt{t_c-t} -\sqrt{t_c-t_0}}{ \sqrt{t_c-t} +\sqrt{t_c-t_0}},
\end{align}
 where $t_c$ is the kinematic cutoff in timelike momentum transfer for
 a particle production threshold and $t_0$ is a parameter that may be
 chosen to improve the series convergence.
Inverting this relation and expanding as a power series in $z$ about $Q^2=-t_0$ yields
\begin{align}
 Q^2+t_0 = 4 (t_c-t_0) \sum_{k=1}^\infty k z^k.
 \label{eq:Q2toz}
\end{align}
A general series in $Q^2$ may therefore be converted to an expansion in $z$ and $t_0$
 by first converting powers of $Q^2$ to those of $Q^2+t_0$ and $t_0$,
\begin{align}
 Q^{2m} &= \big( (Q^2+t_0) -t_0 \big)^m
 = \sum_{n=0}^{m} \left( \begin{array}{c} m \\ n \end{array} \right) (Q^2+t_0)^n (-t_0)^{m-n}
\end{align}
 and then using Eq.~(\ref{eq:Q2toz}) to convert powers of $Q^2+t_0$ into a power series in $z$.
Following this procedure, the LECs that appear as prefactors for powers of $Q^2$
 are instead assembled into expressions related to the coefficients of the $z$ expansion,
\begin{align}
 F\big(z(Q^2)\big) = \sum_k a_k z^k.
 \label{eq:zexp}
\end{align}

%The transformation to take an expansion in $Q^2$ to an expansion in $z$
% is possible because the expression in Eq.~(\ref{eq:Q2toz}) starts at $O(z^1)$.
%The inverse transformation, taking a power series expansion of $z$ to an expansion in $Q^2$,
% is not tractible since all higher powers of $z$ generate terms of order $O\big((Q^2)^0\big)$.
When transforming the expansion in $Q^2$ into an expansion in $z$,
 the one expansion parameter $Q^2$ is replaced instead by two new parameters, $z$ and $t_0$.
The parameter $t_0$ is arbitrary and can be chosen to be small (or 0) based on preference.


To demonstrate this, take $\alpha=\sqrt{(t_c-t)/(t_c-t_0)}$ and expand
\begin{align}
 z = \frac{\alpha-1}{\alpha+1} = 1 - 2 \sum_k (-\alpha)^k,
\end{align}
 then expanding $\alpha^k$ in $t$ yields
\begin{align}
 \alpha^k &= \biggr(\frac{ t_c}{ t_c -t_0} \biggr)^{k/2} \biggr( 1 -\frac{t}{t_c} \biggr)^{k/2}
 \nonumber\\
 &= \sum_{n=0} \biggr(\begin{array}{c} k \\ n \end{array} \biggr)
 \biggr(\frac{ t_c}{ t_c -t_0} \biggr)^{k/2} \biggr( -\frac{t}{2t_c} \biggr)^{n}.
\end{align}
The term that appears with $n=0$ is a $t$-independent term that is multiplied by
 the unknown LEC $a_k$ in the $z$ expansion of Eq.~(\ref{eq:zexp}).

The power series in $z$ fixes the $Q^2$ dependence of the form factor
 by building in the correlations required by analyticity
 between LECs of measurable low powers of $Q^2$
 and in practice unmeasurable higher powers.
Further constraints from perturbative QCD may be used to bound the size of higher-order LECs
 in the expansion.
%Since $|z|<1$ is satisfied for all data in the quasielastic regime,

