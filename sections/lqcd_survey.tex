
\textcolor{red}{[$g_A(0)$ challenges history, excited states]}

Computations of the nucleon axial charge have long been considered
 a vital benchmark for nucleon physics with lattice QCD.
This quantity has been precisely measured with experiments probing neutron beta decay,
 characterized by a weak decay with a low momentum transfer.
Historically, lattice calculations of the axial charge have obtained values
 that were $O(10\%)$ too low, despite significant investments of effort.
This has been the subject of some controversy,
 with more and more sophisticated calculations to scrutinize lattice systematics.
Over time, contamination from excited states was identified as
 an important contributing factor to the systematic deviation.
Many collaborations have opted for high-statistics computations
 to reduce the statistical uncertainty at late times and permit
 multi-exponential fits to the time dependence of the correlation functions.
As time progressed, LQCD results moved closer to the experimental value,
 and now modern calculations are in agreement with experiment
 at the 1\% level~\cite{Kronfeld:2019nfb}.
Modern efforts to compute nucleon matrix elements now target the nucleon
 axial form factor with its four-momentum transfer dependence.

The partially-conserved axial current (PCAC) relation for the correlators,
\begin{align}
 \partial_\mu A^{a,\mu}(x) = 2 m_q P^{a}(x),
 \label{eq:pcac}
\end{align}
 is an exact symmetry in the continuum limit and is a useful
 consistency check for calculations targeting the axial form factor.
The generalization of this relation to the nucleon form factors
 gives the generalized Goldberger-Triemann relation,
\begin{align}
 2 M_N G_A(Q^2) -\frac{Q^2}{2M_N} \widetilde{G}_P(Q^2) = 2 m_q G_{P}(Q^2).
 \label{eq:ggt}
\end{align}
The pion pole dominance ansatz is also checked,
\begin{align}
 \widetilde{G}^{\rm PPD}_P(Q^2) = \frac{4M_N^2}{Q^2+M_\pi^2} G_A(Q^2),
 \label{eq:ppd}
\end{align}
 which is only approximate and is obtained
 by carefully considering the asymptotic behavior of the
 form factors in the double limit $Q^2\to0$ and $m_q\to0$~\cite{Sasaki:2007gw}.
Initial calculations targeting the axial form factor verified the PCAC relation
 for the full correlation functions but found significant \emph{apparent} violations
 of the Generalized Goldberger-Triemann (GGT) relation.
\textcolor{red}{[pion pole dominance estimates]}

\textcolor{red}{[asm still working on below]}
The resolution of the apparent violation of GGT
 is informed by baryon chiral perturbation theory, which suggests that the contaminations
 to the spatial axial, temporal axial, and induced pseudoscalar are very different.
The axial contributions are largely dominated by $N\pi$ excited states
 and are mostly independent of $Q^2$.
The temporal axial current is so afflicted by this contamination that it
 was unable to be successfully fit and was consequently omitted from analyses.
On the other hand, the induced pseudoscalar is driven by the single nucleon contribution and has
 a strong $Q^2$ dependence~\cite{Bar:2018xyi}, with the largest correction at low $Q^2$.
The pseudoscalar contamination is obtained from applying the PCAC relation
 to the contaminations determined for the axial and induced pseudoscalar corrections.

Scrutiny of the excited states again revealed the importance of the contamination in the spectrum.
The excited states contribute unequally in considerations of both
 two-point versus three-point correlation functions and also in
 the contributions to the axial, pseudoscalar, and induced pseudoscalar form factors.
%Three-point correlation functions give a better handle on
% the state energies for a given center of mass momentum than the two-point functions,
% which can lead to a bias if only the two-point functions are used to fix the
% spectrum~\cite{He:2021yvm}.
The axial form factor contribution to the matrix element

%At present, there are a few computations with a complete set of systematic uncertainties.
%The PNDME collaboration released a

\textcolor{red}{[
 Different methods of dealing with excited states.
 FH, 3pt fits, summation, other?
]}

\textcolor{red}{[
 $g_A(Q^2)$ parameterizations.
 $z$ expansion intro, taken from chiral section.
]}

\textcolor{red}{[
$g_A(Q^2)$ from LQCD - 3pt calculations.
both $r_A^2$ and full form factor shape.
derivative methods vs explicit.
highlight importance of providing nondipole parameterization
 coefficients+covariance after chiral-continuum.
]}

\textcolor{red}{[ensemble details of calculations.]}
RQCD and NME with complete error budgets.
actions, ensembles, Mpi.

\textcolor{red}{[more detailed RQCD, NME]}

\textcolor{red}{[connection of Npi with XPT]}
\textcolor{red}{[]}

\textcolor{red}{[]}

Citations for $F_A(Q^2)$ references pulled from NME21, no $g_A$-only references:
\begin{description}
\item[NME 21]~\cite{Park:2021ypf}
\item[RQCD 20]~\cite{Bali:2018qus,RQCD:2019jai} %% 63
\item[ETMC 20]~\cite{Alexandrou:2018sjm,Alexandrou:2019brg,Alexandrou:2020okk} %% 54-56
\item[PACS 18 (erratum)]~\cite{Ishikawa:2018rew,Shintani:2018ozy} %% 60-61 (62 proceedings)
\item[PNDME 17]~\cite{Gupta:2017dwj,Gupta:2018qil,Jang:2019vkm,Jang:2019jkn} %% 6-9
\item[CLS 17]~\cite{Hasan:2017wwt,Hasan:2019noy} %% LHPC? 66-67
\end{description}
Other axial ff effort references:
\begin{description}
\item[new PACS gA(Q2)]~\cite{Ishikawa:2021eut}
\item[Mainz gA(Q2)]~\cite{Djukanovic:2021yqg}
\item[Fermilab Lattice+MILC HISQ gA(0)]~\cite{Lin:2020wko}
\item[CalLat gA(Q2)]~\cite{Meyer:2021vfq}
\end{description}
Other references:
\begin{description}
\item[USQCD white paper]~\cite{Kronfeld:2019nfb}
\item[FLAG 21]~\cite{Aoki:2021kgd}
\item[CalLat excited states $g_A$]~\cite{He:2021yvm}
\item[Ottnad excited states]~\cite{Ottnad:2020qbw}
\item[Baer XPT]~\cite{Bar:2018xyi,Bar:2019igf}
\item[Axial radius from LQCD using XPT]~\cite{Yao:2017fym}
\end{description}

