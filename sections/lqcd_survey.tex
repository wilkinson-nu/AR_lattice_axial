
\textcolor{red}{[$g_A(0)$ challenges history, excited states]}

Computations of the nucleon axial charge have long been considered
 a vital benchmark for nucleon physics with lattice QCD.
This quantity has been precisely measured with experiments probing neutron beta decay,
 characterized by a weak decay with a low momentum transfer.
Historically, lattice calculations of the axial charge have obtained values
 that were $10-20\%$ too low, despite significant investment of effort.
This has been the subject of some controversy,
 with more and more sophisticated calculations to scrutinize lattice systematics.
Over time, contamination from excited states was identified as
 an important contributing factor to the systematic deviation.
Many collaborations have opted for high-statistics computations
 to reduce the statistical uncertainty at late times and permit
 multi-exponential fits to the time dependence of the correlation functions.
As time progressed, LQCD results moved closer to the experimental value,
 and now modern calculations are in agreement with experiment at the 1\% level.

Modern efforts to compute nucleon matrix elements now target the nucleon
 axial form factor with its four-momentum transfer dependence.
Initial calculations revealed a significant apparent violation of the
 generalized Goldberger-Triemann relation, which relates the axial and pseudoscalar
 matrix elements to each other by means of the partially-conserved axial current (PCAC) relation.
Scrutiny of the excited states revealed the importance of nucleon-pion two-particle states
 in the spectrum, which contribute unequally in considerations of both
 two-point versus three-point correlation functions
 and also in matrix elements of an axial current versus a pseudoscalar current.
Even in the case of the axial charge,
 three-point correlation functions give a better handle on
 the state energies for a given center of mass momentum than the two-point functions,
 which can lead to a bias if only the two-point functions are used to fix the
 spectrum~\cite{He:2021yvm}.
In addition, the $G$ parity-violating contribution to the axial and pseudoscalar amplitudes
 at nonzero momentum transfer has a larger amplitude for transitioning
 to nucleon-pion states than to elastically scatter a nucleon,
 which effectively enhances the nucleon-pion contribution in the three-point
 correlators~\cite{Jang:2019vkm}.

\textcolor{red}{[
 Different methods of dealing with excited states.
 FH, 3pt fits, summation, other?
]}

\textcolor{red}{[
 $g_A(Q^2)$ parameterizations.
 $z$ expansion intro, taken from chiral section.
]}

\textcolor{red}{[
$g_A(Q^2)$ from LQCD - 3pt calculations.
both $r_A^2$ and full form factor shape.
derivative methods vs explicit.
highlight importance of providing nondipole parameterization
 coefficients+covariance after chiral-continuum.
]}

\textcolor{red}{[ensemble details of calculations.]}
RQCD and NME with complete error budgets.
actions, ensembles, Mpi.

\textcolor{red}{[more detailed RQCD, NME]}

\textcolor{red}{[connection of Npi with XPT]}
\textcolor{red}{[]}

\textcolor{red}{[]}

Citations for $F_A(Q^2)$ references pulled from NME21, no $g_A$-only references:
\begin{description}
\item[NME 21]~\cite{Park:2021ypf}
\item[RQCD 20]~\cite{Bali:2018qus,RQCD:2019jai} %% 63
\item[ETMC 20]~\cite{Alexandrou:2018sjm,Alexandrou:2019brg,Alexandrou:2020okk} %% 54-56
\item[PACS 18 (erratum)]~\cite{Ishikawa:2018rew,Shintani:2018ozy} %% 60-61 (62 proceedings)
\item[PNDME 17]~\cite{Gupta:2017dwj,Gupta:2018qil,Jang:2019vkm,Jang:2019jkn} %% 6-9
\item[CLS 17]~\cite{Hasan:2017wwt,Hasan:2019noy} %% LHPC? 66-67
\end{description}
Other axial ff effort references:
\begin{description}
\item[new PACS gA(Q2)]~\cite{Ishikawa:2021eut}
\item[Mainz gA(Q2)]~\cite{Djukanovic:2021yqg}
\item[Fermilab Lattice+MILC HISQ gA(0)]~\cite{Lin:2020wko}
\item[CalLat gA(Q2)]~\cite{Meyer:2021vfq}
\end{description}
Other references:
\begin{description}
\item[USQCD white paper]~\cite{Kronfeld:2019nfb}
\item[FLAG 21]~\cite{Aoki:2021kgd}
\item[CalLat excited states $g_A$]~\cite{He:2021yvm}
\item[Ottnad excited states]~\cite{Ottnad:2020qbw}
\item[Baer XPT]~\cite{Bar:2018xyi,Bar:2019igf}
\item[Axial radius from LQCD using XPT]~\cite{Yao:2017fym}
\end{description}

