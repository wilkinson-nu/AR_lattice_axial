
\textcolor{red}{[$g_A(0)$ challenges history, excited states]}

Computations of the nucleon axial charge have long been considered
 a vital benchmark for nucleon physics with lattice QCD.
This quantity has been precisely measured with experiments probing neutron beta decay,
 characterized by a weak decay with a low momentum transfer.
Historically, lattice calculations of the axial charge have obtained values
 that were $O(10\%)$ too low, despite significant investments of effort.
This has been the subject of some controversy,
 with more and more sophisticated calculations to scrutinize lattice systematics.
Over time, contamination from excited states was identified as
 an important contributing factor to the systematic deviation.
Many collaborations have opted for high-statistics computations
 to reduce the statistical uncertainty at late times and permit
 multi-exponential fits to the time dependence of the correlation functions.
As time progressed, LQCD results moved closer to the experimental value,
 and now modern calculations are in agreement with experiment
 at the 1\% level~\cite{Kronfeld:2019nfb}.
\textcolor{red}{[more citations]}
Modern efforts to compute nucleon matrix elements now target the nucleon
 axial form factor with its four-momentum transfer dependence.

To check the accuracy of lattice QCD calculations targeting the axial form factor,
 the validity of the partially-conserved axial current (PCAC) relation,
\begin{align}
 \partial^\mu A^{a}_{\mu}(x) = 2 m_q P^{a}(x),
 \label{eq:pcac}
\end{align}
 has been scrutinized with lattice data.
The PCAC relation is an exact symmetry in the continuum limit.
The generalization of this relation to the nucleon form factors
 yields the generalized Goldberger-Triemann (GGT) relation,
\begin{align}
 2 M_N G_A(Q^2) -\frac{Q^2}{2M_N} \widetilde{G}_P(Q^2) = 2 m_q G_{P}(Q^2),
 \label{eq:ggt}
\end{align}
 which provides orthogonal checks of individual matrix elements
 for the axial and pseudoscalar currents.
The pion pole dominance (PPD) ansatz is also studied,
\begin{align}
 \widetilde{G}^{\rm PPD}_P(Q^2) = \frac{4M_N^2}{Q^2+M_\pi^2} G_A(Q^2),
 \label{eq:ppd}
\end{align}
 which is only approximate even in the continuum and is obtained
 by carefully considering the leading asymptotic behavior of the
 form factors in the double limit $Q^2\to0$ and $m_q\to0$~\cite{Sasaki:2007gw}.

Initial calculations targeting the axial form factor verified the PCAC relation
 for the full correlation functions but found significant \emph{apparent} violations
 of the Generalized Goldberger-Triemann relation, Eq.~\ref{eq:ggt}.
The resolution of the apparent violation of GGT
 is now informed by baryon chiral perturbation theory, which suggests that the
 corrections to the spatial axial, temporal axial, and induced pseudoscalar
 are functionally different and not properly removed.
The axial contributions are largely dominated by $N\pi$ excited states
 with a highly suppressed single nucleon chiral correction.
The correction to the axial current is nearly independent of $Q^2$.
On the other hand, the induced pseudoscalar is driven by the single nucleon contribution and has
 a strong $Q^2$ dependence~\cite{Bar:2018xyi}, with the largest correction at low $Q^2$.
The $N\pi$ contribution in the induced pseudoscalar is highly suppressed by
 an approximate cancellation.
The pseudoscalar contamination is redundant with the axial and induced pseudoscalar
 chiral corrections and can be obtained by application of the PCAC relation.

Scrutiny of the lattice QCD data has demonstrated many of the features
 that were expected from chiral perturbation theory.
The primary excited state contaminations to the axial matrix elements
 were shown to be driven by two specific $N\pi$ states,
 characterized by a transition through an axial current
 of the nucleon state to an $N\pi$ excited state or vise versa~\cite{Jang:2019vkm}.
\textcolor{red}{[is below statement understandable?]}
The momenta of the states with the strongest contribution are those
 where a valence quark diagrams can be drawn containing only a single
 zero-momentum gluon emission, which fixes the relative momenta of the
 quark constituents making up the nucleons and the pion.
These $N\pi$ states were initially expected to be negligible due to a volume suppression
 of the state overlap, which makes them invisible to the two-point functions~\cite{Bar:2016uoj}.
However, the three-point axial matrix element enhances these contributions relative
 to the ground state nucleon matrix element, which is enough to overcome the volume suppression.
As a consequence, analyses that fix the spectrum using the two-point functions alone
 will often miss the important $N\pi$ contamination to the
 axial matrix element~\cite{Jang:2019vkm,He:2021yvm}.

The lattice data also showed deviations as large as $40\%$ from the pion pole
 dominance ansatz at low $Q^2$, where it was expected to
 work best~\cite{Bali:2014nma,Gupta:2017dwj}.
Each $Q^2$ prefered dominant $N\pi$ states with different energies,
 which when neglected across the full range of momentum transfers
 produced a $Q^2$-dependent discrepancy in excess of the $Q^2$ behavior
 expected from the GGT and PPD relations.
Fits to the three-point functions that allowed for the possibility of
 nonnegligible $N\pi$ states are able to constrain the effects of these states,
 removing the contamination and thereby restoring the PPD relation.

\textcolor{red}{[NME]}
\textcolor{red}{[RQCD]}
\textcolor{red}{[ETMC]}

\textcolor{red}{[CLS]}
The CLS collaboration have an axial form factor computation on a
 single ensemble of Wilson clover fermions at physical pion mass~\cite{Hasan:2017wwt}.
This is a methodology paper that focuses on computing the nucleon charges
 and radii by computing derivatives with respect to momenta to the correlation functions.
This method yields the axial radius directly from $Q^2=0$ data,
 which is compared to axial form factor data at nonzero $Q^2$ that
 are parameterized and fit using the traditional three-point method.

\textcolor{red}{[PACS]}
The PACS collaboration has a few computations of the axial form factor
 on large ensembles with Wilson clover fermions at
 physical pion mass~\cite{Ishikawa:2018rew,Shintani:2018ozy}.
They also perform a computation of the axial radius using derivative methods,
 like the CLS collaboration~\cite{Ishikawa:2021eut}.

\begin{itemize}
\item
\textcolor{red}{[ ensemble details of calculations.]}
\item
\textcolor{red}{[
 $g_A(Q^2)$ parameterizations.
 $z$ expansion intro, taken from chiral section.
]}
\item
\textcolor{red}{[
 Different methods of dealing with excited states.
 FH, 3pt fits, summation, other?
]}
\item
\textcolor{red}{[
 axial form factor with full set of coefficients and covariance needed
 in chiral-continuum-FV limit.
 $r_A^2$ reported to connect with pion electroproduction/neutron decay/very low-$Q^2$ applications,
 but not nearly as important for neutrinos.
]}
\item
\textcolor{red}{[ Need an axial ff calculation with explicit $N\pi$-like operators. ]}
\item
\textcolor{red}{[ Will need to verify systematics for large $Q^2$ axial ff,
 but chiral corrections/excited state contaminations are also smallest in this region.
 May be fooled into comfort by breakdown of XPT.
]}
\item
\textcolor{red}{[ form factor slopes at $Q^2=0$ could be good orthogonal constraints.  ]}
\item
\textcolor{red}{[ failure of dipole parameterization with existing LQCD data ]}
\end{itemize}

Citations for $F_A(Q^2)$ references pulled from NME21, no $g_A$-only references:
\begin{description}
\item[NME 21]~\cite{Park:2021ypf}
\item[RQCD 20]~\cite{Bali:2018qus,RQCD:2019jai} %% 63
\item[ETMC 20]~\cite{Alexandrou:2018sjm,Alexandrou:2019brg,Alexandrou:2020okk} %% 54-56
\item[PACS 18 (erratum)]~\cite{Ishikawa:2018rew,Shintani:2018ozy} %% 60-61 (62 proceedings)
\item[PNDME 17]~\cite{Gupta:2017dwj,Gupta:2018qil,Jang:2019vkm,Jang:2019jkn} %% 6-9
\item[CLS 17]~\cite{Hasan:2017wwt,Hasan:2019noy} %% LHPC? 66-67
\end{description}
Other axial ff effort references:
\begin{description}
\item[new PACS gA(Q2)]~\cite{Ishikawa:2021eut}
\item[Mainz gA(Q2)]~\cite{Djukanovic:2021yqg}
\item[Fermilab Lattice+MILC HISQ gA(0)]~\cite{Lin:2020wko}
\item[CalLat gA(Q2)]~\cite{Meyer:2021vfq}
\end{description}
Other references:
\begin{description}
\item[USQCD white paper]~\cite{Kronfeld:2019nfb}
\item[FLAG 21]~\cite{Aoki:2021kgd}
\item[CalLat excited states $g_A$]~\cite{He:2021yvm}
\item[Ottnad excited states]~\cite{Ottnad:2020qbw}
\item[Baer XPT]~\cite{Bar:2018xyi,Bar:2019igf}
\item[Axial radius from LQCD using XPT]~\cite{Yao:2017fym}
\end{description}

