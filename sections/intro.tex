
A major experimental program is underway which seeks to measure
as of yet unknown properties associated with the change of flavor of neutrinos.
In particular, the neutrino mass hierarchy and charge-parity (CP) violating phase
of neutrinos still remain to be measured, with additional focuses on measuring
oscillation parameters with high precision and testing whether the current
three-flavor mixing paradigm is sufficient~\cite{Esteban:2020cvm, ParticleDataGroup:2020ssz}.
These goals introduce stringent requirements on the precision of current and future experiments.
High-intensity beams are required to produce the flux of neutrinos to be able to accumulate the necessary statistics.
Increased statistics place additional burden on our understanding of the systematic uncertainties needed for the experimental program.


Two, billion dollar scale, next-generation experiments designed to meet these experimental constraints
are
%the Deep Underground Neutrino Experiment (DUNE)~\cite{Abi:2020wmh}
DUNE~\cite{Abi:2020wmh}
and the
%Hyper-Kamiokande experiment (Hyper-K)~\cite{Hyper-Kamiokande:2018ofw}.
Hyper-K experiment~\cite{Hyper-Kamiokande:2018ofw}.
DUNE has a broad neutrino energy spectrum with a peak at a neutrino energy of $\approx$2.5 GeV,
but significant contributions between 0.1--10 GeV, over a 1295 km baseline.
Hyper-K has a narrow neutrino energy spectrum peaked at a neutrino energy of $\approx$0.6 GeV, with significant
contributions between 0.1--2 GeV, over a 295 km baseline. Despite their different energies (E) and
baselines (L), both experiments sit at a similar L/E, so probe similar oscillation physics.
%-------------------------------------------------------------------------------
\begin{marginnote}
    \entry{DUNE}{Deep Underground Neutrino Experiment}
    \entry{Hyper-K}{Hyper Kamiokande}
\end{marginnote}
%-------------------------------------------------------------------------------
At the few-GeV energies of interest, neutrino interactions with nucleons have many available interaction channels,
including quasielastic, resonant, and deep inelastic scattering~\cite{zeller12, hayato_review_2014, Mosel:2016cwa, Katori:2016yel, NuSTEC:2017hzk}.
All current and planned experiments use nuclear targets ($^{12}$C--$^{40}$Ar) as the
target material to increase the interaction rate, as well as to avoid serious experimental complications using elementary targets.
The use of nuclear targets significantly complicates the cross-section modeling issues and
associated systematics as intra-nuclear dynamics have a comparable energy scale to
the energy transfers in the neutrino interactions of interest.

A significant challenge impeding progress towards a consistent theoretical description of neutrino-nucleus interactions is the lack of data to benchmark parts of the calculation against. For example, neutrino quasielastic scattering ($\nu_{l} + n \rightarrow l^{-} + p$ or $\bar{\nu}_{l} + p \rightarrow l^{+} + n$) is the simplest of the relevant hard scattering processes, and dominates the neutrino cross section below energies of $\approx$1 GeV. However, modern experiments using nuclear targets are unable to measure it without significant nuclear effects~\cite{garvey_review_2014, NuSTEC:2017hzk}.
Neutrino cross-section models for quasielastic scattering (and other hard-scattering processes) have relied heavily on sparse data from the 1960--1980's from several bubble chamber experiments which used $H_{2}$ or $D_2$ targets~\cite{zeller12, ParticleDataGroup:2020ssz}.
The small neutrino cross section, and relatively weak (by modern standards) accelerator neutrino beams utilized by these early experiments, mean that the available quasielastic event sample on light targets amounts to a few thousand events~\cite{ANL_Barish_1977, BNL_Baker_1981}.
{\color{red} [We could consider adding some comments about pion electroproduction here]}

The sparse data from deuterium bubble chamber experiments do not constrain
the axial form factor precisely.
The popular dipole ansatz has a shape that
is overconstrained by data resulting in an underestimated uncertainty.
Employing a model-independent $z$ expansion parameterization
relaxes the strict shape requirements of the dipole and yields
a more realistic uncertainty that is nearly an order of magnitude larger~\cite{Meyer:2016oeg}.
The axial radius, which is proportional to the slope of the form factor at $Q^2=0$,
has a 50\% uncertainty when estimated from the deuterium scattering data,
or $\approx35\%$ if deuterium scattering and muonic hydrogen are considered
together~\cite{Hill:2017wgb}.
Ideally, the lack of precision in the axial form factor would
be rectified by a modern neutrino scattering experiment.

Safety considerations make it unlikely that new high-statistics bubble chamber experiments using
hydrogen or deuterium will be deployed to fill this crucial gap,
so experimentalists are looking for other ways to access neutrino interactions
with elementary targets as a tool for disambiguating neutrino cross section modeling uncertainties.
One possibility is to use various hydrocarbon targets to subtract the carbon interaction contributions from
the total hydrocarbon event rates, and produce ``on hydrogen'' measurements~\cite{PhysRevD.92.051302, PhysRevD.101.092003, Hamacher-Baumann:2020ogq, DUNE:2021tad}.
These ideas are promising, but typically rely on kinematic tricks that are only relevant for some channels, and it remains to be seen whether the systematic uncertainty associated with modeling the carbon subtraction can be adequately controlled. Such ideas may also be extended to other compound target materials with hydrogen or deuterium components.

In the absence of such an updated experiment,
lattice QCD (LQCD) can provide the missing free nucleon amplitudes
that are otherwise not known at the required precision.
Lattice QCD provides a theoretical alternative for predicting the free nucleon amplitudes directly from the Standard Model (SM) of particle physics, with systematically improvable theoretical uncertainties.
Recently, a benchmark LQCD calculation was achieved in which the nucleon axial charge (the axial form factor at zero momentum transfer) was determined with a 1\% total uncertainty~\cite{Chang:2018uxx}.
Lattice QCD can also provide percent to few-percent level uncertainties for the nucleon quasi-elastic axial form factor with a few-${\rm GeV}^2$ reach in momentum transfer.
Similarly, current tension in the neutron magnetic form factor parameterization, which is roughly half the size of the total axial form factor uncertainty, can be resolved with LQCD calculations.
Such results are anticipated in the next year or so with computing power available in the present near-exascale computing era.

Building upon these critical quantities, more challenging computations can provide information about nucleon
resonant and nonresonant contributions to vector and axial-vector matrix elements,
such as the $\Delta$ or Roper resonance channels, pion-production,
inclusive contributions in the shallow inelastic scattering region,
or deep inelastic scattering parton distribution functions.
Additionally, two-nucleon response functions can be computed which can provide crucial information for our theoretical understanding of important two-body currents which are needed for understanding the neutrino-nucleus cross sections.


Given the present state of the field, in this review, we focus on elastic single-nucleon amplitudes, in which we anticipate the LQCD results will become impactful for the experimental programs in the next year or two.
We begin in Section~\ref{sec:sof} by surveying the existing status and tension in the field for the single-nucleon (quasi-) elastic form factors.
Then, in Section~\ref{sec:lqcd}, after providing a high-level introduction to LQCD, we survey existing results of the axial form factor including the role of the PCAC relation in the calculations as well as use of the $z$-expansion for combining the continuum and physical pion mass extrapolations.
In Section~\ref{sec:impact}, we discuss the potential impact of using LQCD determinations of the axial form factor modeling the $\nu-A$ cross sections.
In Section~\ref{sec:future}, we comment on the most important improvements to be made in LQCD calculations and we conclude in Section~\ref{sec:conclusions}.
