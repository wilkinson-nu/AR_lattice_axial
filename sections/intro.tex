
A major experimental program is underway which seeks to measure
as of yet unknown properties associated with the change of flavor of neutrinos.
In particular, the neutrino mass hierarchy and charge-parity (CP) violating phase
of neutrinos still remain to be measured, with additional focuses on measuring
oscillation parameters with high precision and testing whether the current
three-flavor mixing paradigm is sufficient~\cite{Esteban:2020cvm, ParticleDataGroup:2020ssz}.
These goals introduce stringent requirements on the precision of current and future experiments.
High-intensity beams are required to produce the flux of neutrinos
 to be able to accumulate the necessary statistics.
Increased statistics place additional burden on the systematic uncertainties developed for the experimental program.

Two, billion dollar scale, next-generation experiments designed to meet these experimental constraints
are the Deep Underground Neutrino Experiment (DUNE)~\cite{Abi:2020wmh}
 and the Hyper-Kamiokande experiment (Hyper-K)~\cite{Hyper-Kamiokande:2018ofw}.
DUNE has a broad neutrino energy spectrum with a peak at a neutrino energy of $\approx$2.5 GeV,
but significant contributions between 0.1--10 GeV, over a 1295 km baseline. Hyper-K has a narrow
neutrino energy spectrum peaked at a neutrino energy of $\approx$0.6 GeV, with significant
contributions between 0.1--2 GeV, over a 295 km baseline. Despite their different energies and
baselines, both experiments sit at a similar L/E, so probe similar oscillation physics.
At the few-GeV energies of interest, neutrino interactions with nucleons have many available interaction channels,
 including quasielastic, resonant, and deep inelastic scattering.
In order to increase the interaction cross section,
 large nuclear targets are used in place of elementary targets
 as both target and detection material.
Neutrino interaction cross sections on these large nuclear targets,
 and all of their complications,
 must be understood to make theoretical predictions about neutrino event rates.

Neutrino quasielastic scattering event topology is the simplest of these interactions
 since this interaction makes the assumption of scattering with a single free nucleon.
Despite the simplicity of this interaction,
 intranuclear rescattering of the particles can make the event topology observed
 in the detector more complicated.
Nuclear modeling is required to control these effects,
 which makes isolation of the initial free nucleon interaction more complicated.
In order to access the neutrino interaction cross sections with free nucleons,
 other methods must be employed.

The interaction via a weak current means that scattering cross sections are small,
 and so interaction data with elementary targets is sparse
 (in the case of bubble chamber experiments)
 or subject to uncontrollable model systematics
 (in the case of pion electroproduction experiments).
The dipole ansatz for the axial form factor is overconstrained
 and underestimates the form factor uncertainty by nearly an order of magnitude.
The nucleon axial form factor is poorly constrained from experimental data
 on elementary targets, with a 50\% uncertainty on the axial radius
 obtained from the model-independent $z$ expansion parameterization.
If the axial form factor uncertainty could be decreased,
 tensions in the neutron (?) magnetic form factor are still
 roughly half the size of the total axial form factor uncertainty,
 limiting the precision on the quasielastic neutrino-nucleon cross section.

Restrictions on new high-statistics bubble chamber experiments with
 hydrogen or deuterium fills are not likely due to safety considerations,
 so experiments are looking for other ways to access neutrino interactions
 with elementary targets.
One such way is to use experiments with various hydrocarbon targets
 and subtract away the carbon interaction contributions from
 the total hydrocarbon event rates.

In the absence of an updated scattering experiment on an elementary target,
 lattice QCD (LQCD) could provide the missing free nucleon amplitudes
 that are otherwise not known at the required precision.
The nucleon axial form factor is the first benchmark quantity that could be computed,
 with a few-${\rm GeV}^2$ reach in momentum transfer possible.
More challenging computations could provide information about nucleon
 resonant and nonresonant contributions to axial matrix elements,
 such as the $\Delta$ or Roper resonance channels,
 inclusive contributions in the shallow inelastic scattering region,
 or deep inelastic scattering parton distribution functions.
