LQCD collaborations are able to produce consistent results for benchmark quantities such as $g_{\mathrm{A}}$ with percent level systematic uncertainties, which are in excellent agreement with experimental data.
These results introduce the exciting possibility of LQCD calculations to tackle other quantities for which are not easily experimentally accessible, or for which tensions between measurements, or competing models, exist.
In this review we discussed using LQCD to calculate nucleon form factors as a function of momentum transfer, which are of particular interest to the few-GeV neutrino experimental program.
Important tensions exist in current parameterizations of the vector form factors, but here we focused on the axial form factor, $F_{\mathrm{A}}(Q^2)$, which is of primary importance because current parameterizations of it are simplistic and rely on a handful of low-statistics neutrino--H$_2$ and neutrino--D$_2$ bubble chamber measurements.
It cannot be cleanly measured with existing experiments which use heavier nuclear targets for safety reasons and to increase the event rate, so LQCD offers a novel path to this important quantity.
We have compared $F_{\mathrm{A}}(Q^2)$ calculations from a variety of different LQCD collaborations using different approaches and techniques, and shown them to be in good agreement with each other, but crucially, in poor agreement with the simple dipole model tuned to historic neutrino--D$_2$ data currently relied upon.
Assuming that no systematic effects affecting all of the LQCD calculations are uncovered, this suggests a significant increase of approximately 20\% to the strength of the CCQE scattering channel that dominates the neutrino scattering cross section for $E_{\nu} \lesssim 1$ GeV.
We have demonstrated that these results produce a significant change in the predicted neutrino event spectra for T2K (which has similar considerations to Hyper-K) and DUNE experiments.
Determining the impact on oscillation results would require a full analysis performed by each experimental collaboration, but it is clear that LQCD results for $F_{\mathrm{A}}(Q^2)$ may offer a valuable insight that can clarify aspects of the complex neutrino interaction modeling problem these experiments face.
We additionally discussed a number of ways in which current calculations can be improved and validated, to increase confidence that the LQCD results are not subject to an uncontrolled systematic uncertainty, and to reduce systematics.
Finally, we discussed a number of other quantities which are important to neutrino oscillation experiments in the few-GeV energy regime which LQCD can provide data for, for which there are no experimental solutions, including resonant pion production at higher energy transfers, which is of particular interest to the DUNE experimental program, and insights into nucleon-nucleon correlations.
These possibilities would all work to break degeneracies and overcome experimental challenges that affect current neutrino-nucleus interaction modelling efforts.

%% \cw{Added some very rough bullet points below}
%% \begin{itemize}
%% \item LQCD collaborations are now producing consistent results for a number of benchmark quantities, which are in agreement with experimental data.
%% \item Able to calculate nucleon form factors with current techniques. Results for FA from different collaborations using different approaches are in agreement --- unlikely to be a common missing systematic uncertainty
%% \item LQCD FA results differ from experimental measurements from old nu-H and nu-D bubble chambers. These are important quantities for current neutrino oscillation experiments, but cannot be measured cleanly as all modern neutrino scattering and oscillation experiments use heavy nuclear targets for safety reasons and to increase the event rate.
%% \item We have demonstrated how these results produce a significant change for T2K (Hyper-K) and DUNE experiments. Determining the impact on oscillation results would require a full oscillation analysis performed by each experimental collaboration, but clear that LQCD can reduce uncertainties.
%% \item There are a number of other quantities which are important to neutrino oscillation experiments in the few-GeV energy regime which LQCD can provide data for, for which there are no experimental solutions.
%% \end{itemize}
