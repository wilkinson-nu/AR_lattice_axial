%\cw{Some repetition between this paragraph and Section 2.}
%Computations of the nucleon axial charge have long been considered
% a vital benchmark for nucleon physics with LQCD.
%This quantity has been precisely measured with experiments probing neutron beta decay,
% characterized by a weak decay with a low momentum transfer.
%Historically, lattice calculations of the axial charge have obtained values
% that were $O(10\%)$ too low, despite significant investments of effort.
%This has been the subject of some controversy,
% with more and more sophisticated calculations to scrutinize lattice systematics.
%Over time, contamination from excited states was identified as
% an important contributing factor to the systematic deviation.
%Many collaborations have opted for high-statistics computations
% to reduce the statistical uncertainty at late times and permit
% multi-exponential fits to the time dependence of the correlation functions.
%As time progressed, LQCD results moved closer to the experimental value,
% and now modern calculations are in agreement with experiment
% at the 1\% level~\cite{Kronfeld:2019nfb}.
%\textcolor{red}{[more citations]}

$F_{\rm A}(Q^2)$ can be isolated in LQCD calculations with particular kinematic choices that are not contaminated by the induced pseudoscalar form factor, see for example Ref.~\cite{Gupta:2017dwj}.
Given the challenges in identifying all the sources of systematic uncertainty in the calculation of $g_A$, particularly the excited state contamination, it is prudent to perform cross checks of observables to test for consistency of the results.
%
%Modern efforts to compute nucleon matrix elements now target the nucleon
% axial form factor with its four-momentum transfer dependence.
%Given how difficult the axial charge has been to accurately determine,
% cross checks of observables computed with lattice QCD are vital
% for testing consistency of the results.
%To check the accuracy of LQCD calculations targeting the axial form factor,
% the validity of the PCAC relation,
The validity of the PCAC relation,
\begin{align}
 \partial^\mu A^{a}_{\mu}(x) = 2 m_q P^{a}(x),
 \label{eq:pcac}
\end{align}
provides a complex consistency check.
% has been scrutinized with lattice data.
%is the most reasonable choice for checking consistency.
Here, $A^{a}_\mu$ and $P^{a}$ are the axial and pseudoscalar currents,
 respectively, and $m_q$ is the light quark mass.
The PCAC relation is an exact symmetry in the continuum limit.
The generalization of this relation to the nucleon form factors
 yields the GGT%
 \begin{marginnote}
 \entry{GGT}{Generalized Goldberger-Triemann relation (Equation~\ref{eq:ggt})}
 \end{marginnote}%
 relation,
\begin{align}
 2 m_{\mathrm{N}} F_{\mathrm{A}}(Q^2) -\frac{Q^2}{2m_{\mathrm{N}}} \widetilde{F}_{\mathrm{P}}(Q^2) = 2 m_q F_{\mathrm{P}}(Q^2),
 \label{eq:ggt}
\end{align}
 which provides orthogonal checks of individual matrix elements
 for the axial and pseudoscalar currents.
The axial, induced pseudoscalar, and pseudoscalar form factors of the nucleon
 ($F_{\mathrm{A}}$, $\widetilde{F}_{\mathrm{P}}$, and $F_{\mathrm{P}}$, respectively) appear in this expression,
 and $m_{\mathrm{N}}$ is the nucleon mass.
The PPD ansatz, which is only approximate even in the continuum limit,% is also studied,%
\begin{marginnote}
 \entry{PPD}{Pion pole dominance}
 \end{marginnote}%
\begin{align}
 \widetilde{F}^{\rm PPD}_{\mathrm{P}}(Q^2) = \frac{4m_{\mathrm{N}}^2}{Q^2+m_\pi^2} F_{\mathrm{A}}(Q^2),
 \label{eq:ppd}
\end{align}
% which is only approximate even in the continuum and
is obtained
 by carefully considering the leading asymptotic behavior of the
 form factors in the double limit $Q^2\to0$ and $m_q\to0$~\cite{Sasaki:2007gw}.
%In this expression, $m_\pi$ is the pion mass.

Initial calculations targeting the axial form factor verified the PCAC relation
 for the full correlation functions but found significant \emph{apparent} violations
 of GGT~\cite{Ishikawa:2018rew,Gupta:2017dwj,Bali:2018qus}. The resolution of this apparent violation
 is now informed by baryon $\chi$PT, which suggests that chiral
 and excited state corrections to the spatial axial, temporal axial, and induced pseudoscalar
 are functionally different and not properly removed.
The axial contributions are largely dominated by loop-level $N\pi$ excited states
 with a highly suppressed tree-level correction.
The correction to the axial current is nearly independent of $Q^2$.
On the other hand, corrections to the induced pseudoscalar are
 driven by the tree-level correction and has
 a strong $Q^2$ dependence~\cite{Bar:2018xyi}, with the largest correction at low $Q^2$.
The $N\pi$ loop contribution in the induced pseudoscalar is highly suppressed by
 an approximate cancellation.
The contamination to the pseudoscalar current is redundant with the
 axial and induced pseudoscalar chiral corrections and can be obtained
 by application of the PCAC relation.

Scrutiny of the LQCD data has demonstrated many of the features
 that were expected from chiral perturbation theory.
The primary excited state contaminations to the axial form factor matrix elements
 were shown to be driven by two specific $N\pi$ states,
 characterized by a transition through an axial current
 of the nucleon state to an $N\pi$ excited state or vise versa~\cite{Jang:2019vkm}.
In the language of $\chi$PT, these states contribute to tree-level nucleon-pion graphs with fixed relative momentum~\cite{Bar:2018xyi}.
These $N\pi$ states were initially expected to be negligible due to a volume suppression
 of the state overlap, which makes them invisible to the two-point functions~\cite{Bar:2016uoj}.
However, the three-point axial matrix element enhances these contributions relative
 to the ground state nucleon matrix element, which is enough to overcome the volume suppression.
As a consequence, analyses that fix the spectrum using the two-point functions alone
 will often miss the important $N\pi$ contamination to the
 axial matrix element~\cite{Jang:2019vkm,He:2021yvm}.

The lattice data also showed deviations as large as $40\%$ from the PPD
 ansatz at low $Q^2$, where it was expected to
 work best~\cite{Bali:2014nma,Gupta:2017dwj}.
Each $Q^2$ prefered dominant $N\pi$ states with different energies,
 which when neglected across the full range of momentum transfers
 produced a $Q^2$-dependent discrepancy in excess of the $Q^2$ behavior
 expected from the GGT and PPD relations.
Fits to the three-point functions that allow for the possibility of
 nonnegligible $N\pi$ states are able to constrain the effects of these states,
 removing the contamination and thereby restoring the PPD relation.

Nucleon matrix elements of the temporal axial current have the largest visible excited state contamination~\cite{Jang:2019vkm,RQCD:2019jai} which can be at least qualitatively understood with $\chi$PT~\cite{Bar:2018xyi}.
This has led to new analysis strategies that more carefully deal with the $N\pi$ excited state with pion-pole contributions, yielding minimal violations of PCAC that include such matrix elements, and a better understanding of the impact of these excited states on different matrix elements that are sensitive to either $F_{\rm A}(Q^2)$ or $\tilde{F}_{\rm P}(Q^2)$.

While these results are very encouraging, they are not conclusive in the sense that they require us to impose our theoretical prior on the analysis, rather than the results naturally ``falling out'' from the numerical analysis.
In order to achieve this more stringent validation, the calculations would have to generally be performed at larger source-sink separation times where the correlation functions are all saturated by the ground state, but the noise is exponentially larger.
Given the extreme cost of such a strategy, a better solution would be to implement a variational method that allows for the use of multi-hadron operators that can explicitly remove the $N\pi$ excited states through a diagonalization of the correlation functions~\cite{Blossier:2009kd}.  We will return to this point in Section~\ref{sec:future}.
