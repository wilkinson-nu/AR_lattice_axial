Lattice QCD is a discretized version of QCD, formulated in Euclidean spacetime, in which the quark fields live at the sites of the lattice and the gluon fields live on the links between the sites.  These link fields are given by Wilson-lines
\begin{equation}
U_\mu(x) = \exp\left\{i a\int_0^1 dt A_\mu(x +(1-t)a\hat{\mu}) \right\}
    \approx \exp\left\{i a \bar{A}_\mu(x) \right\}\, ,
\end{equation}
where $A_\mu(x)$ is the gluon field, $a$ is the ``lattice spacing'' and $\bar{A}_\mu(x)$ is the average value of $A_\mu(x)$ over the spacetime interval $[x, x+a\hat{\mu}]$.
This parameterization of the gauge fields allows for the construction of a discretized theory which preserves gauge-invariance~\cite{Wilson:1974sk}, a key property of gauge theories.
For example, the discretized Dirac operator
\begin{equation}\label{eq:naive_fermions}
\bar{\psi}(x)\g_\mu D_\mu \psi(x) \rightarrow
\bar{\psi}(x)\g_\mu\frac{1}{2a}\left[U_\mu(x)\psi(x+a\hat{\mu}) -U^\dagger_\mu(x)\psi(x-a\hat{\mu}) \right]\, ,
\end{equation}
is invariant under gauge transformations,
\begin{align}
&\psi(x)\rightarrow \Omega(x)\psi(x)\, ,&
&U_\mu(x)\rightarrow \Omega(x)U_\mu(x)\Omega^{-1}(x+a\hat{\mu})\, .&
\end{align}
Of note, the transformation of the link field maintains gauge invariance for the combindation of the $\bar{\psi}(x)$ and $\psi(x\pm a\hat{\mu})$ fields.


In the continuum, the gluon action-density is given by the product of field strength tensors, which are gauge-covariant curl's of the gauge potential
\begin{align}
&\mathcal{L}_G = \frac{1}{2g^2}\textrm{Tr}\left[G_{\mu\nu} G_{\mu\nu}\right]\, &
&G_{\mu\nu} = \partial_\mu A_\nu - \partial_\nu A_\mu +i [A_\mu, A_\nu]\, ,&
\end{align}
where $g$ is the gauge coupling.
When constructing the discretized gluon-action, it is therefore natural to use objects which encode this curl of the gauge potential.  The simplest such object is referred to as a ``plaquette'' and given by
\begin{equation}
\hspace{-1.25in}\Umunu \hspace{-0.65in}
    =U_{\mu\nu}(x)
    =U_\mu(x)U_\nu(x+a\hat{\mu}) U^\dagger_\mu(x+a\hat{\nu}) U^\dagger_\nu(x)\, .
\end{equation}
One can then show that the Wilson gauge-action reduces to the continuum action plus irrelevant (higher dimensional) operators which vanish in the continuum limit
\begin{align}\label{eq:gluon_action}
S_G(U) &= \beta \sum_{n=x/a} \sum_{\mu<\nu}
    \textrm{Re}\left[ 1 - \frac{1}{N_c} \textrm{Tr} \left[U_{\mu\nu}(n) \right]\right]
\nonumber\\&=
    \frac{\beta}{2N_c} a^4 \sum_{n=x/a,\mu,\nu} \frac{1}{2}
    \textrm{Tr} \left[ G_{\mu\nu}(n)G_{\mu\nu}(n)\right]
    +\mathrm{O}(a^6)\, ,
    & \rightarrow \beta = \frac{2N_c}{g^2}\, .
\end{align}
The continuum limit, which is the assymptotically large $Q^2$ region, is therefore approached as $\beta\rightarrow\infty$ where $g(Q^2)\rightarrow 0$.

There are many choices one can make in constructing the discretized lattice action.
Provided continuum QCD is recovered as $a\rightarrow0$, each choice is valid.
This is known as the universality of the continuum limit, with each choice only varying at finite lattice spacing.
Deviations from QCD, which arise at finite $a$, are often called \textit{discretization corrections} or \textit{scaling violations}.
That all lattice actions reduce to QCD as $a\rightarrow0$ is known as the universality of the continuum limit.  It is a property which can be proved in perturbation theory but must be established numerically given the non-perturbative nature of QCD.
For sufficiently small lattice spacings, one can use effective field theory (EFT) to construct a continuum theory that encodes the discretization effects in a tower of higher dimensional operators.  This is known as the Symanzik EFT for lattice actions~\cite{Symanzik:1983dc,Symanzik:1983gh}.
One interesting example involves the violation of Lorentz Symmetry at finite lattice spacing: in the Symanzik EFT, the operators which encode this Lorentz violation scale as $a^2$ with respect to the operators which survive the continuum limit, and thus, Lorentz symmetry is an accidental symmetry of the continuum limit.  It is not respected at any finite lattice spacing, but the measurable consequences vanish as $a^2$ for sufficiently small lattice spacing.


The inclusion of quark fields adds more variety of lattice actions.
One main complication for fermions is that the ``naive discretization'', equation~\eqref{eq:naive_fermions}, leads to a well-known doubling problem in which, instead of a single fermion, one has $2^d$ doubler fermions, or poles in the propagator in $d$ dimensions.
In particular, it is challenging to remove these doublers without breaking chiral symmetry with the lattice regulator, an issue known as the Nielsen-Ninomiya No Go Theorem~\cite{Nielsen:1981hk,Nielsen:1980rz,Nielsen:1981xu}.
There are four commonly used fermion discretization schemes that deal with this no-go theorem in different ways:
\begin{itemize}[leftmargin=*]
\item Staggered, or Kogut-Susskind fermions~\addcite{KS}, split the four components of the fermion spinor onto different corners of a local hyper-cube, reducing the doublers by a factor of four.  A fourth root of the fermion determinant is then used to simulate a single fermion flavor.  Staggered fermions retain a remnant chiral symmetry, protecting the quark mass from additive mass renormalization.  They are the least expensive numerically to simulate, and thus state of the art computations now include six lattice spacings to control the continuum extrapolation.  Splitting the spinor components to different lattice sites complicates the Dirac structure of the theory, mixing it with the spacetime symmetries, and complicating the construction of hadrons with spin, such as nucleons~\addcite{Aaron knows some refs}.

\item Clover-Wilson fermions add two irrelevant dimension-5 operators to the action.  The first is a double derivative operator (the Wilson opeator) that gives a mass to the doublers that scales as $1/a$, thus lifting all but one of the doublers to be a heavy state that decouples in the continuum limit.  The double derivative operator explicitly breaks chiral symmetry causing an $1/a$ additive shift to the quark mass, thus requiring fine-tuning to retain the small up and down quark masses.
The second operator is the clover operator, $a c_{SW} \bar{q} \sigma_{\mu\nu} G_{\mu\nu} q$.  The $c_{SW}$ coefficient~\addcite{Csw} can be tuned to remove the residual $\mathrm{O}(a)$ chiral symmetry breaking discretization effects from the spectrum~\addcite{O(a) papers}.  Removing such $\mathrm{O}(a)$ effects from all quantities, such as nucleon matrix elements and form factors, requires the tuning of additional irrelevant operators to remove such scaling violations from currents and other operators.  This procedure is well understood, and routinely carried out, but is also laborious and time consuming.  Nevertheless, since Wilson type fermions respect all other symmetries and they are relatively inexpensive to simulate, they are the most widely used in the literature.

\item Twisted-mass fermions~\addcite{tm papers} are a variant of Wilson or clover-Wilson fermions where the approximate $SU(2)$ flavor symmetry of the light quarks is used to add a mass term proportional to $i\gamma_5 \tau_3$.  This mass terms is protected from additive mass renormalization, and so if one works at \textit{maximal twist}, such that the regular mass term just cancels the $1/a$ mass term from the Wilson operator, the theory is automatically $\mathrm{O}(a)$ improved, meaning the leading $\mathrm{O}(a)$ chiral symmetry breaking corrections that normally arise for Wilson fermions are removed.
Twisted mass fermions break isospin symmetry at finite lattice spacing, causing some complications now that LQCD results are precise enough to require isospin breaking corrections from $m_d-m_u$ and QED to be compared with experiment.

\item The fourth most common discretization are Domain Wall Fermions (DWF)~\addcite{}, which introduce a fifth dimension to the theory with unit links (the gluons are not dynamic in the fifth dimension).  The left and right handed fermions are bound to opposite sides of the fifth dimension.  The overlap of these left and right modes gives rise to an explicit chiral symmetry breaking that is exponentially suppressed by the extent of the fifth dimension.  For sufficiently small chiral symmetry breaking (large $L_5$), DWF are also automatically $\mathrm{O}(a)$ improved.
While very desirable, DWF are numerically more expensive to simulate, both because of the extra fifth dimension and also because the algorithmic speed up offered by multi-grid, which works tremendously for clover-Wilson fermions~\addcite{Kate + Balint Titan/Summit}, is not yet flushed out for DWF~\addcite{Boyle and QUDA people}.

\item A final common variant of action is one in which the fermion discretization used in the generation of the gauge fields (the sea quarks) and the action used when generating quark propagators (the valence quarks) are different: this is known as a \textit{mixed action}~\cite{Renner:2004ck}.
The most common reason to use such an action is to take advantage of numerically less expensive methods to generate the configurations while retaining good chiral symmetry properties of the valence quarks, which is known to suppress chiral symmetry breaking effects from the sea-quarks~\cite{Bar:2002nr,Bar:2005tu,Tiburzi:2005is,Chen:2007ug}.


\end{itemize}
There is an extensive literature on LQCD and its application to computing properties of nucleons.
For reviews concerning nucleon structure, see Refs.~\addcite{}.
For in depth introductions to lattice QCD, see the text books~\cite{Smit:2002ug,DeGrand:2006zz,Gattringer:2010zz}.






\addcomment{thoughts for rest of intro to LQCD}
\begin{itemize}

\item refer to reviews of LQCD and text books, Smitt, Gatringer and Lang, DeTar and DeGrand

\item full calculation requires extrapolation to the continuum limit, physical pion mass, infinite volume

\item 2pt function

\item {\color{red}[move to next section?]} extra challenges for FF: light pions = more noise, (light pions more expensive cause larger L, larger condition number, and more statistics needed); finite L limits accessible kinematic points - particularly challenging for extracting slope of form factors at small $Q^2$ - FF parameterizations can show non-trivial structure in between kinematically accessible points - leads to ideas to compute slope of FF as well as FF...

\end{itemize}
