Lattice QCD is a discretized version of QCD, formulated in Euclidean spacetime, in which the quark fields live at the sites of the lattice and the gluon fields live on the links between the sites.  These link fields are given by Wilson-lines
\begin{equation}
U_\mu(x) = \exp\left\{i a\int_0^1 dt A_\mu(x +(1-t)a\hat{\mu}) \right\}
    \approx \exp\left\{i a \bar{A}_\mu(x) \right\}\, ,
\end{equation}
where $A_\mu(x)$ is the gluon field, $a$ is the ``lattice spacing'' and $\bar{A}_\mu(x)$ is the average value of $A_\mu(x)$ over the spacetime interval $[x, x+a\hat{\mu}]$.
This parameterization of the gauge fields allows for the construction of a discretized theory which preserves gauge-invariance~\addcite{Wilson}, a key property of gauge theories.
For example, the discretized Dirac operator
\begin{equation}
\bar{\psi}(x)\g_\mu D_\mu \psi(x) \rightarrow
\bar{\psi}(x)\g_\mu\frac{1}{2a}\left[U_\mu(x)\psi(x+a\hat{\mu}) -U^\dagger_\mu(x)\psi(x-a\hat{\mu}) \right]\, ,
\end{equation}
is invariant under gauge transformations,
\begin{align}
&\psi(x)\rightarrow \Omega(x)\psi(x)\, ,&
&U_\mu(x)\rightarrow \Omega(x)U_\mu(x)\Omega^{-1}(x+a\hat{\mu})\, .&
\end{align}
In the continuum, the gluon action-density is given by product of field strength tensors, which are gauge-covariant curl's of the gauge potential
\begin{align}
&\mathcal{L}_G = \frac{1}{2g^2}\textrm{Tr}\left[G_{\mu\nu} G_{\mu\nu}\right]\, &
&G_{\mu\nu} = \partial_\mu A_\nu - \partial_\nu A_\mu +i [A_\mu, A_\nu]\, ,&
\end{align}
where $g$ is the gauge coupling.
When constructing the discretized gluon-action, it is therefore natural to use objects which encode this curl of the gauge potential.  The simplest such object is referred to as a ``plaquette'' and given by
\begin{equation}
\hspace{-1.25in}\Umunu \hspace{-0.65in}
    =U_{\mu\nu}(x)
    =U_\mu(x)U_\nu(x+a\hat{\mu}) U^\dagger_\mu(x+a\hat{\nu}) U^\dagger_\nu(x)\, .
\end{equation}
One can then show that the Wilson gauge-action reduces to the continuum action plus irrelevant operators which vanish in the continuum limit
\begin{align}
S_G(U) &= \beta \sum_{n=x/a} \sum_{\mu<\nu}
    \textrm{Re}\left[ 1 - \frac{1}{N_c} \textrm{Tr} \left[U_{\mu\nu}(n) \right]\right]
\nonumber\\&=
    \frac{\beta}{2N_c} a^4 \sum_{n=x/a,\mu,\nu} \frac{1}{2}
    \textrm{Tr} \left[ G_{\mu\nu}(n)G_{\mu\nu}(n)\right]
    +\mathrm{O}(a^6)\, ,
    & \rightarrow \beta = \frac{2N_c}{g^2}\, .
\end{align}
The continuum limit, which is the assymptotically large $Q^2$ region, is therefore approached as $\beta\rightarrow\infty$ where $g(Q^2)\rightarrow 0$.



\addcomment{thoughts for rest of intro to LQCD}
\begin{itemize}

\item many choices of discretized action

\item many choices of fermions

\item Symanzik shows us all choices reduce to QCD close to the continuum, plus irrelevant operators

\item advantages and disadvantages of different choices, cost for $\mathrm{O}(a^2)$ improved

\item full calculation requires extrapolation to the continuum limit, physical pion mass, infinite volume

\item extra challenges for FF: light pions = more noise, (light pions more expensive cause larger L, larger condition number, and more statistics needed); finite L limits accessible kinematic points - particularly challenging for extracting slope of form factors at small $Q^2$ - FF parameterizations can show non-trivial structure in between kinematically accessible points - leads to ideas to compute slope of FF as well as FF...

\end{itemize}
