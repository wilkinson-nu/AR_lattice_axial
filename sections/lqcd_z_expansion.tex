
Lattice data are computed with nonzero lattice spacings and typically at unphysical pion masses.
To control for the effects of these systematics,
 chiral perturbation theory and a Symanzik effective theory are typically employed.
These connect the lattice results to the physical point by means
 of a perturbative expansion in parameters such as the pion mass or lattice spacing.
Powers of these expansion parameters come with some low energy constants (LECs)
 that must be fit to the lattice data over various ensembles with different parameters.
With the LECs in hand, extrapolation to the continuum physical point
 may be completed and results obtained for QCD.

These extrapolations are known for the axial charge,
 but chiral expansions in $Q^2$ have not been formulated as of yet.
In their place, one might choose to instead parameterize the form factor as
 a Taylor expansion in $Q^2$,
\begin{align}
 F(Q^2) = \sum_{k=0} \ell_k (Q^2)^k,
\end{align}
 where the LECs $\ell_k$ have dependence on the pion mass and lattice spacing.
This form factor expansion only has limited validity at $Q^2$ close to zero,
 failing as higher orders become important.

The $z$ expansion is a conformal mapping from the four-momentum transfer squared $Q^2$
 to a small expansion parameter $z$, parameterized by the relation
\begin{align}
 z(t=-Q^2;t_0,t_c) = \frac{\sqrt{t_c-t} -\sqrt{t_c-t_0}}{ \sqrt{t_c-t} +\sqrt{t_c-t_0}},
\end{align}
 where $t_c$ is the kinematic cutoff in timelike momentum transfer for
 a particle production threshold and $t_0$ is a parameter (typically negative)
 that may be chosen to improve the series convergence.
Inverting this relation and expanding as a power series in $z$ about $Q^2=-t_0$ yields
\begin{align}
 Q^2+t_0 = 4 (t_c-t_0) \sum_{k=1}^\infty k z^k.
 \label{eq:Q2toz}
\end{align}
A general series in $Q^2$ may therefore be converted to an expansion in $z$ and $t_0$
 by first converting powers of $Q^2$ to those of $Q^2+t_0$ and $t_0$ using binomial theorem,
\begin{align}
 Q^{2m} &= \big( (Q^2+t_0) -t_0 \big)^m
 = \sum_{n=0}^{m} \left( \begin{array}{c} m \\ n \end{array} \right) (Q^2+t_0)^n (-t_0)^{m-n}
\end{align}
 and then using Eq.~(\ref{eq:Q2toz}) to convert powers of $Q^2+t_0$ into a power series in $z$.
Following this procedure, the LECs that appear as prefactors for powers of $Q^2$
 are instead assembled into expressions related to the coefficients of the $z$ expansion,
\begin{align}
 F\big(z(Q^2)\big) = \sum_k a_k z^k.
 \label{eq:zexp}
\end{align}
To obtain an expansion to order $O\big(z^{k}\big)$,
 a starting expansion of order $O\big((Q^2)^{k}\big)$ is sufficient.

When transforming the expansion in $Q^2$ into an expansion in $z$,
 the one expansion parameter $Q^2$ is replaced instead by two new parameters, $z$ and $t_0$.
The parameter $t_0$ is arbitrary and can be chosen to be small (or 0) based on preference.
The relative size of $t_0$ and $z$ are inversely related,
 so by making $t_0$ larger a higher expansion order in $t_0$
 may be traded for a lower expansion order in $z$.
However, $t_0$ must still be kept small enough that the expansions in $t_0$ converge.

For the inverse transformation to take an expansion in $z$ to an expansion in $Q^2$,
 the second expansion parameter that appears is $\sqrt{t_c/(t_c-t_0)}$ instead of $t_0$.
To demonstrate this, take $\alpha=\sqrt{(t_c-t)/(t_c-t_0)}$ and expand
\begin{align}
 z = \frac{\alpha-1}{\alpha+1} = 1 - 2 \sum_k (-\alpha)^k,
\end{align}
 then expanding $\alpha^k$ in $t$ yields
\begin{align}
 \alpha^k &= \biggr(\frac{ t_c}{ t_c -t_0} \biggr)^{k/2} \biggr( 1 -\frac{t}{t_c} \biggr)^{k/2}
 = \sum_{n=0} \biggr(\begin{array}{c} k \\ n \end{array} \biggr)
 \biggr(\frac{ t_c}{ t_c -t_0} \biggr)^{k/2} \biggr( -\frac{t}{2t_c} \biggr)^{n}.
\end{align}
This transformation instead prefers \emph{larger} $-t_0$ such that
 $\sqrt{t_c/(t_c-t_0)}<1$ and does not converge for $t_0=0$.
The term that appears with $n=0$ is a $t$-independent term that is multiplied by
 the unknown LEC $a_k$ in the $z$ expansion of Eq.~(\ref{eq:zexp}).

The power series in $z$ fixes the $Q^2$ dependence of the form factor
 by building in the correlations required by analyticity
 between LECs of measurable low powers of $Q^2$
 and in practice unmeasurable higher powers.
Further constraints from perturbative QCD may be used to bound the size of higher-order LECs
 in the $z$ expansion.
