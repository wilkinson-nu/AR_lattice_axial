
Lattice data are computed with nonzero lattice spacings and typically at unphysical pion masses.
To control for the effects of these systematics,
 chiral perturbation theory and a Symanzik effective theory are typically employed.
These connect the lattice results to the physical point by means
 of a perturbative expansion in parameters such as the pion mass or lattice spacing.
Powers of these expansion parameters come with some low energy constants (LECs)
 that must be fit to the lattice data over various ensembles with different parameters.
With the LECs in hand, extrapolation to the continuum and physical pion mass
 may be completed and results obtained for QCD.

These extrapolations are known for the axial charge,
 but heavy baryon chiral expansions with $Q^2$ dependence are largely unexplored.
In their place, one might choose to instead expand the form factor as
 a power series in $Q^2$ with low energy constant (LEC) prefactors $\ell_k$,
\begin{align}
 F(Q^2) = \sum_{k=0} \ell_k (Q^2)^k,
\end{align}
 where the $\ell_k$ have some unknown dependence on the pion mass and lattice spacing.
This form factor expansion will only has limited validity for $Q^2$ close to zero.
One approach to circumvent this failure mode is to appeal to the analytic structure of QCD,
 performing a conformal mapping to a obtain a new small expansion parameter.
This parameterization, referred to as the $z$ expansion~\cite{Bhattacharya:2011ah},
 has been utilized for decades in meson flavor physics~\cite{Okubo:1971jf}
 and is a standard feature in modern LQCD calculations of meson form factors
 for determining CKM matrix elements.

The $z$ expansion takes the four-momentum transfer squared $Q^2$
 to a small expansion parameter $z$, using the relation
\begin{align}
 z(t=-Q^2;t_0,t_c) = \frac{\sqrt{t_c-t} -\sqrt{t_c-t_0}}{ \sqrt{t_c-t} +\sqrt{t_c-t_0}}.
\end{align}
Here, $t_c$ is no larger than the particle production threshold in timelike momentum transfer
 and $t_0$ is a parameter (typically negative) that
 may be chosen to improve the series convergence.
Inverting this relation and expanding as a power series in $z$ about $Q^2=-t_0$ ($z=0$) yields
\begin{align}
 \frac{Q^2+t_0}{t_c-t_0} = 4 \sum_{k=1}^\infty k z^k.
 \label{eq:Q2toz}
\end{align}
Following this procedure, the dimensionful LECs that appear as prefactors for powers of $Q^2$
 are instead assembled into expressions related to the dimensionless
 coefficients of the $z$ expansion,
\begin{align}
 F\big(z(Q^2)\big) = \sum_{k=0}^\infty a_k z^k.
 \label{eq:zexp}
\end{align}
%
For the inverse transformation to take an expansion in $z$ to an expansion in $Q^2$,
 the expansion is again cast in terms of $x=(Q^2+t_0)/(t_c-t_0)$, like in Eq.~(\ref{eq:Q2toz}).
Then the expression for $z$ at small $x$ is
\begin{align}
 (1+x)^{1/2} -1 = \frac{x}{2}
 -4\sum_{k=2}^\infty \frac{(2k-3)!}{k!(k-2)!} \biggr( -\frac{x}{4} \biggr)^{k},
 &\quad
 z = \frac{1}{x} \big( (1+x)^{1/2} -1 \big)^2,
 \label{eq:ztoQ2}
\end{align}
 which starts at $O(x)$.
For large $Q^2$, the relation is instead expanded in terms of $x^{-1}$.

A general series in $Q^2$ may therefore be converted to a double expansion in $z$ and $t_0$
 by first converting powers of $Q^2$ to those of $Q^2+t_0$ and $t_0$ using binomial theorem,
\begin{align}
 Q^{2m} &= \big( (Q^2+t_0) -t_0 \big)^m
 = (t_c-t_0)^m
 \sum_{n=0}^{m} \left( \begin{array}{c} m \\ n \end{array} \right)
 \biggr( \frac{Q^2+t_0}{t_c-t_0} \biggr)^n \biggr( \frac{-t_0}{t_c-t_0} \biggr)^{m-n}
\end{align}
 and then substituting Eq.~(\ref{eq:Q2toz}) to convert powers of $Q^2+t_0$
 into powers of $z$.
All dependence on the dimension is absorbed into powers of $t_c-t_0 \propto M_\pi^2$.
The relative weight of the expansion parameters may be adjusted by changing
 the value of $t_0$, giving some modicum of freedom over the expansion order.

The most recent multi-ensemble lattice QCD publications with computations
 of the axial form factor~\cite{Park:2021ypf,RQCD:2019jai}
 have treated the $z$ expansion coefficients as the relevant LECs
 and fit to these coefficients with a chiral-continuum extrapolation.
No attempt was made to connect these LECs to those obtained
 from chiral expansions with explicit $Q^2$ dependence.
Instead, observables such as $r_A^2$ and $g_A$
 were fit to separate chiral-continuum extrapolations.
Application of the formulae in this section exposes the relationships
 between LECs for powers of $Q^2$ to the $z$ expansion coefficients.
\textcolor{red}{[Citations with chiral expansions that expose $Q^2$ dependence?]}

