{\color{red}items to comment on}
\begin{itemize}
\item won't be performing precision calculations at 2-3 fm
\item variational with pi-N operators
\item this enables Breit-Frame and improved single-nucleon
\item domain-decomposed idea of Giusti et al
\end{itemize}


Precise nucleon form factors for (quasi)elastic scattering are of the most immediate concern
 for neutrino oscillation experiments and simultaneously one of the easiest
 targets for lattice QCD in regards to nucleon interactions.
However, processes with higher energy transfers are also a cause for concern,
 especially when considering the higher-energy neutrino flux at DUNE (see Figure~\ref{fig:dune_impact}).
These larger energy transfers can access other fundamentally different interaction topologies,
 such as resonant or nonresonant pion production mechanisms,
 nuclear responses with correlated nucleon pairs,
 or scattering off partons within nucleons.
In principle, all of these interaction mechanisms are accessible to lattice QCD,
 though with varying degrees of difficulty.
Given the discrepancy between lattice axial form factor data and experimental constraints,
 it is not unreasonable to expect other interaction mechanisms have similar discrepancies
 between theory and observation.
These interaction types are more challenging to extract or even inaccessible to
 experimental measurements.
Appeals to model assumptions may give some handle for missing quantities,
 but are also subject to unquantifiable systematic effects.

Calculations that access the combined resonant and nonresonant scattering amplitudes
 are the most similar to those of elastic scattering,
 where a current induces a transition of the nucleon to a multiparticle final state.
Only stable final states, such as nucleon-pion states or other multiparticle states,
 are obtained as eigenstates of the Hamiltonian.
Resonance properties are indirectly probed by converting the observed spectrum,
 with its power-law finite volume corrections, to a scattering phase shift in infinite volume.
These power-law finite volume corrections are in contrast with exponentially-damped
 corrections obtained for single-particle states.
The multiparticle states make up a dense spectrum that arises from
 states where individual hadrons move with different discrete lattice momenta.

The main difficulties in these calculations stem the challenges of extracting
 information about many excited states rather than a single ground state.
Empirical evidence from dedicated computations demonstrates that these multiparticle states
 are in practice difficult to quantify without interpolating operators constructed to
 closely resemble the states they are intended to identify,
 both in the number of quarks and antiquarks and also the individual
 momenta of the quark-level components~\textcolor{red}{[cite JLab]}.
The necessary interpolating operators for extracting multiparticle states often have
 unfavorable combinatoric factors multiplying the number of terms to account for
 permutations of quark lines, changing momenta, varying time ranges, or the like.
Some form of approximate all-to-all method is likely necessary
 to compute amplitudes of nucleon resonant and nonresonant transitions
 to states with pions.

The aforementioned issues are, at least in part, circumvented by applying approximate
 all-to-all techniques, for instance distillation~\textcolor{red}{[citations]}.
Though the overhead for these methods is large,
 these techniques enable sophisticated calculations with large bases
 of interpolating operators including multiparticle interpolators.
Using an all-to-all method specifically avoids the need to produce sequential
 quark propagator inversions that are used in traditional three-point correlator functions,
 which are the source of many unfavorable combinatoric factors and are
 costly when enumerating all possible quark line combinations.
Though the lowest resonances will be accessible with these techniques,
 the higher-mass resonances will be more technically difficult
 due to the increased density of states with the energy transfers involved.
Using these propagator data, a calculation of the Roper resonance contribution
 has the secondary consequence of offering a handle on removing excited
 state contamination from a calculation of the quasielastic form factors.
\textcolor{red}{[summarizing statement or two about current status]}

To access higher resonance states in the so-called shallow inelastic scattering (SIS) regime,
 lattice calculations will likely be more successful using other methods
 that access the inclusive nucleon scattering amplitude to hadronic states.
A four-point function calculation on the lattice can obtain a discrete sampling
 of the scattering amplitude as a function of the center-of-mass energy,
 convolved with weights that are exponential in the energy and time separation.
Applying an inverse transformation to obtain the scattering amplitude
 is an ill-posed problem, but initial calculations using techniques
 such as Backus-Gilbert or moments have shown promise.
\textcolor{red}{[more detail, references, check wording]}

Calculations of two-nucleon matrix elements provide key insights
 about the correlations between nucleons inside of a nuclear medium,
 a vital ingredient for construction of an effective theory
 of neutrino interactions with nuclear targets.
%% heavier than physical Mpi => NPLQCD/HALQCD controversy
First efforts have been made to compute deuteron and dineutron scattering phase shifts
 at unphysically heavy pion masses.
There is some controversy about whether the deuteron forms a
 bound state at these higher pion masses.~\textcolor{red}{[citations, more content]}.

%% gA(Q2) on D2 and nuclear ET
Future calculations of matrix elements for currents inserted between
 two-nucleon states could provide direct information about the LEC inputs to nuclear
 models~\cite{Drischler:2019xuo}
 or by providing direct comparisons of neutrino interaction matrix elements in deuterium.
\textcolor{red}{[more detail needed on ET LECs]}
%% overlap of energy scales of NN vs NNpi
\textcolor{red}{[the following is likely too much detail]}
Deuterium corrections from nuclear models were assumed to be strong only at low momentum transfer
 and energy-independent in the reanalysis of deuterium bubble chamber data~\cite{Meyer:2016oeg},
 despite the inability of these corrections to account for the theory-data discrepancies.
A direct lattice QCD computation of these effects would isolate the effect,
 either by definitively attributing the discrepancy to deuterium effects
 or by implicating the other systematics corrections.
