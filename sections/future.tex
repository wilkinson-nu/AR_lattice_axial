
Precise nucleon form factors for (quasi)elastic scattering are of the most immediate concern
 for neutrino oscillation experiments and simultaneously one of the easiest
 targets for lattice QCD in regards to nucleon interactions.
However, processes with higher energy transfers are also a cause for concern,
 especially when considering the higher-energy neutrino flux at DUNE.
Given the discrepancy between lattice axial form factor data and experimental constraints,
 it is not unreasonable to expect other interaction mechanisms are similarly discrepant.
These interaction types are more challenging for or even inaccessible to
 experimental measurements.
Appeals to model assumptions may give some handle for missing quantities,
 but are also subject to unquantifiable systematic effects.
Larger energy transfers can access other fundamentally different interaction topologies,
 such as resonant or nonresonant pion production mechanisms,
 nuclear responses with correlated nucleon pairs,
 or scattering off partons within nucleons.
In principle, all of these interaction mechanisms are accessible to lattice QCD,
 though with varying degrees of difficulty.

Calculations that access resonant scattering matrix elements are the most similar
 to calculations with elastic scattering.
The main source of difficulty in these calculations stem from trying to extract
 information about the excited states rather than the ground state.
A dense spectrum arises from states that contain multiple particles
 moving with several different discrete lattice momenta.
Empirical evidence from dedicated computations demonstrates that these multiparticle states are
 in practice difficult to quantify without interpolating operators that closely
 resemble those states in their construction~\textcolor{red}{[cite JLab]}.
Not only are these states dense, but they have a strong power-law dependence on the
 lattice spatial volume, as compared to exponentially-damped corrections
 for single-particle states, leading to nontrivial corrections to the state masses.
In addition, the necessary correlation functions for extracting these states often have
 unfavorable combinatoric factors multiplying the number of terms to account for
 permutations of quark lines, changing momenta, varying time ranges, or the like.

%To access information about the resonant scattering of neutrinos,
% the aforementioned issues will need to be overcome by some
% combination of clever algorithms and computational power.
Some form of approximate all-to-all method is likely necessary
 to compute amplitudes of nucleon resonant and nonresonant transitions
 to states with pions.
The aforementioned issues are, at least in part, circumvented by applying approximate
 all-to-all techniques, for instance distillation~\textcolor{red}{[citations]}.
Though the overhead for these methods is large,
 these techniques enable sophisticated calculations with large bases
 of interpolating operators including multiparticle interpolators.
Using an all-to-all method specifically avoids producing the sequential
 quark propagator inversions, like in traditional three-point correlator functions,
 that become costly when enumerating all possible quark line combinations
 and especially in the case of multiparticle operators.
The higher-mass resonances will be more technically difficult
 due to the increased density of states with the energy transfers involved.
Using these propagator data, a calculation of the Roper resonance contribution
 has the secondary consequence of offering a handle on removing excited
 state contamination from a calculation of the quasielastic form factors.
\textcolor{red}{[summarizing statement or two about current status]}

Calculations of two-nucleon matrix elements provide key insights
 about the correlations between nucleons inside of a nuclear medium,
 a vital ingredient for construction of an effective theory
 of neutrino interactions with nuclear targets.
%% heavier than physical Mpi => NPLQCD/HALQCD controversy
First efforts have been made to compute deuteron and dineutron scattering phase shifts
 at unphysically heavy pion masses.
There is some controversy about whether the deuteron forms a
 bound state at these higher pion masses.~\textcolor{red}{[citations, more content]}.
%% gA(Q2) on D2
Once this controversy is cleared,
 matrix elements of currents inserted between two-nucleon states could
 provide information about the strength of nuclear corrections in deuterium.
These corrections were assumed to be strong only at low momentum transfer
 and energy-independent in the deuterium bubble chamber data reanalysis~\cite{Meyer:2016oeg},
 for which an uncertainty inflation was needed to account for observed theory-data discrepancies.
%% energy scales of NNpi


