LQCD results are computed at finite lattice spacing, in a finite volume and typically (for nucleons) at pion masses heavier than nature.
Even when results at the physical pion mass are available, results at heavier pion masses can help improve the overall precision of the final result, provided the extrapolation to the physical pion mass is under control.
EFT is extensively used to guide the extrapolation to the physical point.
Symanzik EFT~\cite{Symanzik:1983dc,Symanzik:1983gh} provides a continuum description of a discretized lattice action that incorporates the lattice spacing corrections through a tower of increasingly irrelevant operators.
$\chi$PT provides a description of perturbative light quark mass corrections to pions~\cite{Gasser:1984gg} and nucleons~\cite{Jenkins:1990jv,Bernard:1995dp} in a systematic expansion about the chiral limit and low momentum, also through a series of increasingly irrelevant operators.
$\chi$PT can be readily extended to systematically include discretization effects by constructing the EFT from the Symanzik EFT, which includes QCD as the leading Lagrangian and discretization effects through the higher dimensional operators~\cite{Sharpe:1998xm}.
$\chi$PT can also be extended to incorporate the boundary conditions of the finite volume and describe the finite-volume corrections which scale as $m_\pi^n e^{-m_\pi L}$ for $m_\pi L \gtrsim3.5$ with the power $n$ depending upon the particular quantity of interest~\cite{Gasser:1986vb}.

While these EFTs provide a complete description of the physics at low momentum and sufficiently light pion masses, they come with LECs that capture the short-distance (L)QCD physics which are a priori unknown, given the non-perturbative nature of QCD.%
%-------------------------------------------------------------------------------
\begin{marginnote}
\entry{LEC(s)}{Low energy constant(s)}
\end{marginnote}%
%-------------------------------------------------------------------------------
The LECs describing the QCD pion mass and momentum dependence can be determined both from comparing with experimental data as well as LQCD results, while the LECs describing the discretization corrections can only be determined by comparing with LQCD results as these LECs are specific to a given lattice action.

The range of validity of $\chi$PT seems to be at best $m_\pi\lesssim300$~MeV~\cite{Beane:2004ks,Walker-Loud:2008rui} with some indications the convergence is troubled at the physical pion mass~\cite{Walker-Loud:2019cif,Drischler:2019xuo}.
The reach in $Q^2$ will likely have a similar upper bound, though this has not been examined in nearly the same detail as the pion mass expansion.
For many quantities, a power series expansion in $m_\pi$ or $m_\pi^2$ may be perfectly sufficient, combined with sufficiently precise results at the physical pion mass, to guide the interpolation to the physical point, as was observed in detail for $g_A$~\cite{Chang:2018uxx}.
For example, one can consider an expansion of the form factor as a power-series in $Q^2$ with coefficients that capture the pion mass and lattice spacing dependence
\begin{align}\label{eq:F_Q_power}
F(Q^2) = \sum_{k=0} f_k(m_\pi, a) Q^{2k},
\end{align}
where these prefactors will depend upon the LECs of QCD and the discretized lattice action.
This form factor expansion only has limited validity for $Q^2$ close to zero.
One approach to circumvent this failure mode is to appeal to the analytic structure of QCD,
performing a conformal mapping to a obtain a new small expansion parameter.
This parameterization, referred to as the $z$ expansion~\cite{Bhattacharya:2011ah},
has been utilized for decades in meson flavor physics~\cite{Okubo:1971jf}
and is a standard feature in modern LQCD calculations of meson form factors
for determining CKM matrix elements.




\iffalse
\bigskip
Lattice data are computed with nonzero lattice spacings and typically at unphysical pion masses.
To control for the effects of these systematics,
 chiral perturbation theory and a Symanzik effective theory are typically employed.
These connect the lattice results to the physical point by means
 of a perturbative expansion in parameters such as the pion mass or lattice spacing.
Powers of these expansion parameters come with some LECs
 that must be fit to the lattice data over various ensembles with different parameters.
With the LECs in hand, extrapolation to the continuum and physical pion mass
 may be completed and results obtained for QCD.

These extrapolations are known for the axial charge,
 but heavy baryon chiral expansions with $Q^2$ dependence are largely unexplored.
In their place, one might choose to instead expand the form factor as
 a power series in $Q^2$ with LEC prefactors $\ell_k$,
\begin{align}
 F(Q^2) = \sum_{k=0} \ell_k (Q^2)^k,
\end{align}
 where the $\ell_k$ have some unknown dependence on the pion mass and lattice spacing.
This form factor expansion will only has limited validity for $Q^2$ close to zero.
One approach to circumvent this failure mode is to appeal to the analytic structure of QCD and
 perform a conformal mapping to a obtain a new small expansion parameter.
This parameterization, referred to as the $z$ expansion~\cite{Bhattacharya:2011ah},
 has been utilized for decades in meson flavor physics~\cite{Okubo:1971jf}
 and is a standard feature in modern LQCD calculations of meson form factors
 for determining CKM matrix elements.
\fi



The $z$ expansion takes the four-momentum transfer squared $Q^2$
 to a small expansion parameter $z$, using the relation
\begin{align}
 z(t=-Q^2;t_0,t_c) = \frac{\sqrt{t_c-t} -\sqrt{t_c-t_0}}{ \sqrt{t_c-t} +\sqrt{t_c-t_0}}.
\end{align}
Here, $t_c$ is no larger than the particle production threshold in timelike momentum transfer
 and $t_0$ is a parameter (typically negative) that
 may be chosen to improve the series convergence.
Inverting this relation and expanding as a power series in $z$ about $Q^2=-t_0$ ($z=0$) yields
\begin{align}
x =  \frac{Q^2+t_0}{t_c-t_0} = 4 \sum_{k=1}^\infty k z^k.
 \label{eq:Q2toz}
\end{align}
Following this procedure, the dimensionful LECs that appear as prefactors for powers of $Q^2$
 are instead assembled into expressions related to the dimensionless
 coefficients of the $z$ expansion (to avoid confusion with the lattice spacing, we label the coefficients as $b_k$),
\begin{align}
 F\big(z(Q^2)\big) = \sum_{k=0}^\infty b_k z^k.
 \label{eq:zexp}
\end{align}
%
For the inverse transformation to take an expansion in $z$ to an expansion in $Q^2$,
 the expansion is again cast in terms of $x$
like in Eq.~(\ref{eq:Q2toz}).
Then the expression for $z$ at small $x$ is
\begin{align}
 (1+x)^{1/2} -1 = \frac{x}{2}
 -4\sum_{k=2}^\infty \frac{(2k-3)!}{k!(k-2)!} \biggr( -\frac{x}{4} \biggr)^{k},
 &\quad
 z = \frac{1}{x} \big( (1+x)^{1/2} -1 \big)^2,
 \label{eq:ztoQ2}
\end{align}
 which starts at $O(x)$.
For large $Q^2$, the relation is instead expanded in terms of $x^{-1}$.

A general series in $Q^2$ may therefore be converted to a double expansion in $z$ and $t_c-t_0$
 by first converting powers of $Q^2$ to those of $Q^2+t_0$ and $t_0$ using the  binomial theorem,
\begin{align}
 Q^{2m} &= \big( (Q^2+t_0) -t_0 \big)^m
 = (t_c-t_0)^m
 \sum_{n=0}^{m} \left( \begin{array}{c} m \\ n \end{array} \right)
 x^n
 %\biggr( \frac{Q^2+t_0}{t_c-t_0} \biggr)^n
 \biggr( \frac{-t_0}{t_c-t_0} \biggr)^{m-n}
\end{align}
 and then substituting Eq.~(\ref{eq:Q2toz}) to convert powers of $Q^2+t_0$
 into powers of $z$.
All dependence on the dimension is absorbed into powers of $t_c-t_0 \propto m_\pi^2$.
The relative weight of the expansion parameters may be adjusted by changing
 the value of $t_0$, giving some modicum of freedom over the expansion order.

The most recent multi-ensemble LQCD publications with computations
 of the axial form factor~\cite{Park:2021ypf,RQCD:2019jai}
 have treated the $z$ expansion coefficients as the relevant LECs
 and fit to these coefficients with a chiral-continuum extrapolation.
No attempt was made to connect these LECs to those obtained
 from chiral expansions with explicit $Q^2$ dependence.
Instead, observables such as $r_{\mathrm{A}}^2$ and $g_{\mathrm{A}}$
 were fit to separate chiral-continuum extrapolations.
Application of the formulae in this section exposes the relationships
 between LECs for powers of $Q^2$ to the $z$ expansion coefficients.
As a simple example, consider the $Q^4$ expansion of equation~\eqref{eq:F_Q_power} with
the a truncation at the leading chiral and discretization corrections
\begin{align}
f_0 &= c_0 + \ell_0 m_\pi^2 + d_0 a^2\, ,
\nonumber\\
f_1 &= c_1 + \ell_1 m_\pi^2 + d_1 a^2\, ,
\nonumber\\
f_2 &= c_2 + \ell_2 m_\pi^2 + d_2 a^2\, ,
\label{eq:lecampi}
\end{align}
where $c_k$, $\ell_k$ and $d_k$ are LECs describing the pion mass and lattice spacing dependence.
For the axial form factor, $c_0$ and $c_1$ are related to the axial charge and radius in the chiral limit
\begin{align}
&c_0 = \lim_{m_\pi\rightarrow0} g_A\, ,&
&c_1 = -\lim_{m_\pi\rightarrow0} \frac{g_A r_A^2}{6}\, .&
\end{align}
Then the $z$ expansion coefficients that appear in equation~(\ref{eq:zexp})
 expressed in terms of these LECs are
\newcommand{\tctza}{\ensuremath{(t_c-t_0)}}
\newcommand{\tctzb}{\ensuremath{\Big(\frac{-t_0}{t_c-t_0}\Big)}}
\begin{align}\label{eq:z_coeff_xpt}
b_0 &= f_0 +\tctza \tctzb f_1 +\tctza^2 \tctzb^2 f_2 +O\big(\tctza^3\big),
\nonumber\\
b_1 &= 4 \tctza f_1 +8 \tctza^2 \tctzb f_2 +O\big(\tctza^3\big),
\nonumber\\
b_2 &= 8 \tctza f_1 +16 \tctza^2 \Big(1 -\frac{t_0}{t_c-t_0}\Big) f_2 +O\big(\tctza^3\big).
\end{align}
The leading lattice spacing and pion mass dependence from
 the LECs for the $Q^2$ expansion may be made manifest
 by substituting their expresions from equation~(\ref{eq:lecampi}).
Higher-order corrections to the $\chi$PT expressions may be checked
 by peforming fits to the LECs of the $z$ expansion and comparing those
 to fits to the LECs of the $Q^2$ expansion.

When the $z$ expansion coefficients are expressed as in equation~\eqref{eq:z_coeff_xpt}, a simultaneous fit of the LQCD form factor data can be performed across multiple ensembles at different lattice spacings and pions masses, with coefficients connected to the EFTs that are valid for sufficiently small pion mass and $Q^2$.
Similarly, the finite volume corrections can also be incorporated in this analysis: what we have called the LECs in the $f_k(m_\pi,a)$ coefficients are not just LECs, but they also encode the non-analytic contributions arising from virtual pion loops, and thus, they can be evaluated in finite as well as infinite volume.
