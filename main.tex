% template.tex, dated April 5 2013
% This is a template file for Annual Reviews 1 column Journals
%
% Compilation using ar-1col.cls' - version 1.0, Aptara Inc.
% (c) 2013 AR
%
% Steps to compile: latex latex latex
%
% For tracking purposes => this is v1.0 - Apr. 2013

\documentclass{ar-1col}
\usepackage{url}
\usepackage[numbers,sort&compress]{natbib}

%% added packages
\usepackage{xspace}
\usepackage{graphicx}
\usepackage{amssymb}
\usepackage{enumitem}
%\usepackage[top=2cm,bottom=3cm]{geometry}
\usepackage[svgnames]{xcolor}
% hyper-ref prevents the document from compiling on the arXiv - it is
% included in some other `usepackage` and that generates a conflict.
% Will attempt to use it differently, hope for a better outcome.
%\usepackage[bookmarks=false]{hyperref}
\usepackage[bookmarks=false,colorlinks=true,linkcolor=DarkBlue,citecolor=DarkBlue]{hyperref}
\usepackage{subfig}
%\usepackage{subcaption}
\usepackage{rotating}
\usepackage{units}
\usepackage{amsmath}
\usepackage{bm}     % nice math bold italics
\usepackage{lineno}
\usepackage{listings}
\usepackage{lmodern} % suppress font warnings
\usepackage[normalem]{ulem}
\usepackage[section]{placeins}
\usepackage{hepunits}
\usepackage{hepparticles}
\usepackage{cancel}
\usepackage{hepnames}
%\usepackage{epstopdf}
\usepackage{mathtools}
\usepackage[capitalise]{cleveref}
\usepackage{braket}
\usepackage{slashed}
%\usepackage{xcolor}
\usepackage{multirow}

\usepackage[Export]{adjustbox}

\providecommand{\cw}[1]{{\color{purple} Callum: #1}}
\newcommand\citeneeded{[{\color{blue} \underline{CITATION NEEDED}}]}

\def\g{\gamma}
\def\O{\Omega}
\def\D{\Delta}

\def\Dslash{D\hskip-0.65em /}


%\newcommand{\Umunu}{\includegraphics[width=0.5\textwidth,fbox=1pt,valign=c]{plots/plaquette}}
\newcommand{\Umunu}{\includegraphics[width=0.6\textwidth,valign=c]{plots/plaquette}}

\setcounter{secnumdepth}{4}
\def\addcite#1{{\color{red}CITE[#1]}}
\def\addcomment#1{{\color{red}COMMENT: #1}}
\def\new#1{{\color{blue}#1}}
\def\asm#1{{\color{red}#1}}
\def\done#1{{\color{gray}#1}}


% Metadata Information
\jname{Xxxx. Xxx. Xxx. Xxx.}
\jvol{AA}
\jyear{2022}
\doi{10.1146/((please add article doi))}


% Document starts
\begin{document}

% Page header
\markboth{A. Meyer, A. Walker-Loud, C. Wilkinson}{LQCD relevance for the few-GeV neutrino program}

% Title
\title{Status of Lattice QCD Determination of Nucleon Form Factors
 and their Relevance for the Few-GeV Neutrino Program}

%Authors, affiliations address.
\author{Aaron S. Meyer$^1$,
Andr\'{e} Walker-Loud$^2$,
Callum Wilkinson$^3$
\affil{$^1$Department of Physics, University of California, Berkeley, CA, 94720, USA}
\affil{$^2$Nuclear Science Division, Lawrence Berkeley National Laboratory, Berkeley, CA, 94720, USA}
\affil{$^3$Physics Division, Lawrence Berkeley National Laboratory, Berkeley, CA, 94720, USA}
}

\begin{abstract}
Neutrino-nucleon interactions provide the dominant contribution to neutrino-nucleus cross sections, which are critical inputs to billion-dollar experimental efforts aimed at measuring as of yet unknown neutrino oscillation parameters.
The neutrino-nucleon interactions are difficult to measure experimentally and current parameterizations
rely on low-statistics measurements from a handful of historic measurements to inform a number of nucleon form factors and other key quantities.
Lattice QCD can be used to determine these interactions directly from the Standard Model with fully quantified theoretical uncertainties.
Recent lattice QCD results of the $g_{\mathrm{A}}$ are in excellent agreement with experimental data,
offering hope that soon, results for the (quasi-)elastic nucleon form factors will be available.
We review the status of the field and lattice QCD results for the nucleon axial form factor, $F_{\mathrm{A}}(Q^2)$, a major source of uncertainty in neutrino-nucleon interaction models for $E_{\nu} \lesssim 1$ GeV.
Results from different lattice calculations are in good agreement with each other, but collectively, they are in poor agreement with existing models of $F_{\mathrm{A}}(Q^2)$.
We discuss the potential impact of these lattice QCD results for current and future neutrino oscillation experiments.
We describe a road map solidify confidence in the lattice results and we discuss future calculations of more complicated processes, such as the resonant neutrino-nucleon reactions which are important to neutrino oscillation experiments in the few-GeV energy regime.
\end{abstract}

%Keywords, etc.
\begin{keywords}
Neutrino Oscillations, Nucleon Form Factors, Lattice QCD
%keywords, separated by comma, no full stop, lowercase
\end{keywords}

\maketitle

%Table of Contents
\tableofcontents


% ------------------------------------------------------------------------------
% Intoroduction
\section{Introduction\label{sec:intro}}
A major experimental program is underway which seeks to measure
as of yet unknown properties associated with the change of flavor of neutrinos.
In particular, the neutrino mass hierarchy and%
\begin{marginnote}
\entry{CP}{charge-parity}
\end{marginnote}%
CP violating phase
of neutrinos still remain to be measured, with additional focuses on measuring
oscillation parameters with high precision and testing whether the current
three-flavor mixing paradigm is sufficient~\cite{Esteban:2020cvm, ParticleDataGroup:2020ssz}.
These goals introduce stringent requirements on the precision of current and future experiments.
High-intensity beams are required to produce the flux of neutrinos to be able to accumulate the necessary statistics.
Increased statistics place additional burden on our understanding of the systematic uncertainties needed for the experimental program.


Two, billion dollar scale, next-generation experiments designed to meet these experimental constraints
are
%the Deep Underground Neutrino Experiment (DUNE)~\cite{Abi:2020wmh}
DUNE~\cite{Abi:2020wmh}
and the
%Hyper-Kamiokande experiment (Hyper-K)~\cite{Hyper-Kamiokande:2018ofw}.
Hyper-K experiment~\cite{Hyper-Kamiokande:2018ofw}.
DUNE has a broad neutrino energy spectrum with a peak at a neutrino energy of $\approx$2.5 GeV,
but significant contributions between 0.1--10 GeV, over a 1295 km baseline.
Hyper-K has a narrow neutrino energy spectrum peaked at a neutrino energy of $\approx$0.6 GeV, with significant
contributions between 0.1--2 GeV, over a 295 km baseline. Despite their different energies (E) and
baselines (L), both experiments sit at a similar L/E, so probe similar oscillation physics.%
%-------------------------------------------------------------------------------
\begin{marginnote}
    \entry{DUNE}{Deep Underground Neutrino Experiment}
    \entry{Hyper-K}{Hyper Kamiokande}
\end{marginnote}%
%-------------------------------------------------------------------------------
At the few-GeV energies of interest, neutrino interactions with nucleons have many available interaction channels,
including quasielastic, resonant, and deep inelastic scattering~\cite{zeller12, hayato_review_2014, Mosel:2016cwa, Katori:2016yel, NuSTEC:2017hzk}.
All current and planned experiments use nuclear targets ($^{12}$C--$^{40}$Ar) as the
target material to increase the interaction rate, as well as to avoid serious experimental complications using elementary targets.
The use of nuclear targets significantly complicates the cross-section modeling issues and
associated systematics as intra-nuclear dynamics have a comparable energy scale to
the energy transfers in the neutrino interactions of interest.

A significant challenge impeding progress towards a consistent theoretical description of neutrino-nucleus interactions is the lack of data to benchmark parts of the calculation against. For example, neutrino quasielastic scattering ($\nu_{l} + n \rightarrow l^{-} + p$ or $\bar{\nu}_{l} + p \rightarrow l^{+} + n$) is the simplest of the relevant hard scattering processes, and dominates the neutrino cross section below energies of $\approx$1 GeV. However, modern experiments using nuclear targets are unable to measure it without significant nuclear effects~\cite{garvey_review_2014, NuSTEC:2017hzk}.
Neutrino cross-section models for quasielastic scattering (and other hard-scattering processes) have relied heavily on sparse data from the 1960--1980's from several bubble chamber experiments which used H$_{2}$ or D$_2$ targets~\cite{zeller12, ParticleDataGroup:2020ssz}.
The small neutrino cross section, and relatively weak (by modern standards) accelerator neutrino beams utilized by these early experiments, mean that the available quasielastic event sample on light targets amounts to a few thousand events~\cite{ANL_Barish_1977, BNL_Fanourakis_1980, BNL_Baker_1981, Kitagaki:1983px, Allasia:1990uy}.%
\footnote{Some constraints on the axial form factor have been obtained from fits
 to pion electroproduction data.
The fits are parameterized by a low energy theory that is valid in the chiral
 ($M_\pi\to0$) limit and close to threshold.
These data will be ignored in this review.
For more details, we refer the reader to Refs.~\cite{Bernard:1993bq,Bernard:2001rs}.
}

The sparse data from deuterium bubble-chamber experiments do not constrain
the axial form factor precisely.
The popular dipole ansatz has a shape that
is overconstrained by data resulting in an underestimated uncertainty.
Employing a model-independent $z$-expansion parameterization
relaxes the strict shape requirements of the dipole and yields
a more realistic uncertainty that is nearly an order of magnitude larger~\cite{Meyer:2016oeg}.
The axial radius, which is proportional to the slope of the form factor at $Q^2=0$,
has a 50\% uncertainty when estimated from the deuterium scattering data,
or $\approx35\%$ if deuterium scattering and muonic hydrogen are considered
together~\cite{Hill:2017wgb}.
Ideally, the lack of precision in the axial form factor would
be rectified by a modern neutrino-scattering experiment.

Safety considerations make it unlikely that new high-statistics bubble-chamber experiments using
hydrogen or deuterium will be deployed to fill this crucial gap,
so experimentalists are looking for other ways to access neutrino interactions
with elementary targets as a tool for disambiguating neutrino cross-section modeling uncertainties.
One possibility is to use various hydrocarbon targets to subtract the carbon interaction contributions from
the total hydrocarbon event rates, and produce ``on hydrogen'' measurements~\cite{PhysRevD.92.051302, PhysRevD.101.092003, Hamacher-Baumann:2020ogq, DUNE:2021tad}.
These ideas are promising, but typically rely on kinematic tricks that are only relevant for some channels, and it remains to be seen whether the systematic uncertainty associated with modeling the carbon subtraction can be adequately controlled. Such ideas may also be extended to other compound target materials with hydrogen or deuterium components.

In the absence of such an updated experiment,%
\begin{marginnote}
\entry{LQCD}{Lattice quantum chromodynamics}
\end{marginnote}%
LQCD can provide the missing free nucleon amplitudes
that are otherwise not known at the required precision.
LQCD provides a theoretical alternative for predicting the free nucleon amplitudes directly from the Standard Model of particle physics, with systematically improvable theoretical uncertainties.
Recently, a benchmark LQCD calculation was achieved in which the nucleon axial charge (the axial form factor at zero momentum transfer) was determined with a 1\% total uncertainty~\cite{Chang:2018uxx}.
LQCD can also provide percent to few-percent level uncertainties for the nucleon quasielastic axial form factor with a few-${\rm GeV}^2$ reach in momentum transfer.
Similarly, current tension in the neutron magnetic form factor parameterization, which is roughly half the size of the total axial form factor uncertainty, can be resolved with LQCD calculations.
Such results are anticipated in the next year or so with computing power available in the present near-exascale computing era.

Building upon these critical quantities, more challenging computations can provide information about nucleon
resonant and nonresonant contributions to vector and axial-vector matrix elements,
such as the $\Delta$ or Roper resonance channels, pion-production,
inclusive contributions in the shallow inelastic scattering region,
or deep inelastic scattering parton distribution functions.
Additionally, two-nucleon response functions can be computed which can provide crucial information for our theoretical understanding of important two-body currents which are needed for understanding the neutrino-nucleus cross sections.


Given the present state of the field, in this review, we focus on elastic single-nucleon amplitudes, in which we anticipate the LQCD results will become impactful for the experimental programs in the next year or two.
We begin in Section~\ref{sec:sof} by surveying the existing status and tension in the field for the single-nucleon (quasi-) elastic form factors.
Then, in Section~\ref{sec:lqcd}, after providing a high-level introduction to LQCD, we survey existing results of the axial form factor including the role of the%
\begin{marginnote}
 \entry{PCAC} {Partially-conserved} axial current
\end{marginnote}%
PCAC relation in the calculations as well as use of the $z$-expansion for combining the continuum and physical pion mass extrapolations.
In Section~\ref{sec:impact}, we discuss the potential impact of using LQCD determinations of the axial form factor when modeling neutrino-nucleus cross sections.
In Section~\ref{sec:future}, we comment on the most important improvements to be made in LQCD calculations and we conclude in Section~\ref{sec:conclusions}.



%-------------------------------------------------------------------------------
% State of the field
\section{Status of single nucleon (quasi-) elastic form factors\label{sec:sof}}
For charged current quasi-elastic scattering of a neutrino with a free nucleon,
the neutron ($|n\rangle$) to proton ($\langle p|$)
interaction is described by a $V-A$ weak interaction, given at the quark level by
$\bar{u}\gamma_\mu(1- \gamma_5)d$ (or its conjugate for proton to neutron), with the nucleon level amplitude at four-momentum transfer $Q^2 = -q^2$ parameterized by
\begin{align}\label{eq:nucleon_ff}
\langle p | V^\mu | n \rangle
    &= \bar{U}_p(p+q) \Big[
        F_1^+(q^2) \gamma^\mu
        +\frac{i}{2M} F_2^+(q^2) \sigma^{\mu\nu} q_\nu
    \Big] U_n(p),
\nonumber\\
\langle p | A^\mu | n \rangle
    &= \bar{U}_p(p+q) \Big[
        F_A^+(q^2) \gamma^\mu \gamma_5
        +\frac{1}{M} F_P^+(q^2) q^\mu \gamma_5
    \Big] U_n(p)\, .
\end{align}
The isovector, vector form factors, $F_1^+$ and $F_2^+$, can be precisely estimated from electron-nucleon scattering data.
Electron-proton and electron-neutron scattering are sensitive to the isoscalar, $F_{1,2}^s$ and isovector, $F_{1,2}^3$ form factors.  After isolating $F_{1,2}^3$, approximate isospin symmetry can be used to relate these $\tau_3$ form factors to the charged $\tau_+$ form factors of equation~\eqref{eq:nucleon_ff}: in the isospin limit, $\langle p| \bar{u}\, \Gamma u - \bar{d}\, \Gamma d |p\rangle = \langle p| \bar{u}\, \Gamma d |n\rangle$ for dirac structure $\Gamma$.

%------------------------------------------------------------------------------
% proton magnetic FF
\begin{figure}
 \centering
 \includegraphics[width=0.7\textwidth]{plots/proton_magnetic-standalone.pdf}
\caption{
Proton magnetic form factor normalized by a reference dipole ansatz
with a dipole mass of $0.84~{\rm GeV}$.
The proton-only fit to a $z$ expansion by Borah {\it et al.}~\cite{Borah:2020gte}
and the BBBA05 parameterization~\cite{Bradford:2006yz} are shown.
\label{fig:protonmagneticff}
}
\end{figure}
%------------------------------------------------------------------------------

Even though the overall uncertainty of the electron-neutron form factors is larger than those of the proton, there is a significant tension in existing parameterizations of the proton magnetic form factor
shown in Figure~\ref{fig:protonmagneticff}.
Two different parameterizations of the form factor, normalized by the dipole are displayed.
The Bradford, Bodek, Budd and Arrington parameterization from Neutrino-Nucleus Interaction 2005 Workshop (BBBA05)~\cite{Bradford:2006yz} are displayed as the lower (blue) band with a solid mean value.
A more recent z-expansion parameterization from Borah {\it et al.}~\cite{Borah:2020gte} is displayed by the upper (black) band with a dashed mean value.
The tension is significant over all $Q^2 > 0$, at the level of several percent,
including significant disagreement in the slope of the form factor at $Q^2 = 0$.
Of the nucleon form factor calculations from lattice QCD,
the vector form factors are also the most mature,
exhibiting no obvious tensions with experimental determinations
of the vector form factors at their current level of precision.
A percent level calculation of the form factor $Q^2$ behavior combined with a direct calculation of the slope of the magnetic form factor would provide useful insight about this tension or could discriminate
between the two parameterizations.


The nucleon axial form factor has had a much more complicated past than the vector form factors in LQCD.
The axial charge, the value of the axial form factor at $Q^2=0$, is a key benchmark for LQCD and is precisely known
from neutron decay experiments~\addcite{add gA refs}.
LQCD calculations of the axial charge have historically been low compared to experiment~\addcite{some LQCD review}, and the discrepancy has been the topic of some controversy.
It is now understood that the treatment of excited state systematics is the main culprit for this discrepancy~\cite{Bar:2017kxh,Ottnad:2020qbw,Aoki:2021kgd}.
With proper control over the excited state contamination, the LQCD calculations are now in good agreement with the experimental value~\addcite{modern values} with one group achieving a sub-percent determination of $g_A$~\cite{Chang:2018uxx,Berkowitz:2018gqe,Walker-Loud:2019cif}.


Armed with the newfound confidence in the control of systematics for the axial charge,
more emphasis is now being put into the form factor at nonzero momentum transfer.
One extremely striking feature of lattice calculations is the strong preference for a slower
fall off of the form factor with increasing $Q^2$ than predicted by phenomenological determinations from experiment.
This preference is consistently reproduced by several lattice collaborations using
independent computation methods, lending more credence to the result.
When integrated over the full range of $Q^2$ to compute the nucleon cross section,
this translates to an enhancement in the free nucleon cross section as large as $30-40\%$
over neutrino energies greater than $1~{\rm GeV}$.
In addition, the precision on the axial form factor uncertainty from lattice QCD
is small enough to be sensitive to the tension between vector form factor parameterizations.


The aforementioned situation with nucleon form factors is depicted in Figure~\ref{fig:nucleonxsec}.
{\color{red}[Andre: would it be useful to make this paragraph a bullet list?]}
The black dot-dashed curve is the default case, which uses the $z$ expansion parameterizations of the vector and axial form factors from Refs.~\cite{Borah:2020gte}~and~\cite{Meyer:2016oeg}, respectively.
The green band shows the uncertainty obtained from the axial form factor alone, and the gray band from the (much smaller) vector form factor uncertainty by itself.
The blue solid curve substitutes the vector form factors of the default choice with the BBBA05 parameterization~\cite{Bradford:2006yz}, taking the uncorrelated uncertainty from the BBBA05 vector form factors only.
The observed tension between the black and blue bands is the result of the tension between proton magnetic form factor parameterizations.
The red dotted curve instead substitutes the axial form factor of the default with a parameterization obtained from lattice QCD and its uncertainty.
The size of the red band with respect to the green band demonstrates the uncertainty reduction from replacing the deuterium scattering axial form factor with one obtained from LQCD, and the significant change in normalization is due to the slower fall off of the axial form factor.
Additionally, the size of the red band demonstrates the relevance of the tension in vector form factors, characterized by the difference between the black and blue curves.


%------------------------------------------------------------------------------
% nu-N cross section
\begin{figure}[hbt!]
 \centering
 \includegraphics[width=0.7\textwidth]{plots/xsec_comparison-standalone.pdf}
\caption{
 Neutrino cross sections on a free neutron, with their uncertainty bands,
 for various choices of parameterization.
 The curves labeled ``BBBA05'' (blue solid line, Ref.~\cite{Bradford:2006yz})
 and ``$z$ exp, vector'' (black dot-dashed line, Ref.~\cite{Borah:2020gte}) use the
 $z$ expansion axial form factor from Ref.~\cite{Meyer:2016oeg},
 with only the uncertainty from the vector form factors plotted
 to highlight the tension between the parameterizations shown in Fig.~\ref{fig:protonmagneticff}.
 The same form factor parameterizations are used for both ``$z$ exp, vector'' and
 ``$z$ exp, $D_{2}$ axial'' (green dot-dashed line)
 but in the latter case the uncertainty band is taken only from
 the axial form factor rather than only from the vector form factor.
 The red dotted line labeled ``$z$ exp, LQCD axial'' is parameterized by
 the vector form factors of Ref.~\cite{Borah:2020gte} with no uncertainty
 and the axial form factor with its uncertainty taken from LQCD.
 \label{fig:nucleonxsec}
}
\end{figure}

In the next section, we review the current status of LQCD calculations of the axial form factor.
We emphasize that LQCD calculations of the electric and magnetic form factors are more mature, but equally important.
We anticipate, given the state of the field, that within a couple years, we will have a complete error budget for the single nucleon (quasi-) elastic form factors from LQCD.






\bigskip\noindent{\color{red}comments below:}
\begin{description}
\item[first $z$ exp vector form factor] \cite{Ye:2017gyb}
 - does not use constraints from muonic hydrogen, many more fit parameters
\item[Vector form factor tensions] \cite{Borah:2020gte}
\item[Muonic hydrogen review] \cite{Hill:2017wgb}
\end{description}

\textcolor{red}{[Most focus on axial, but vector is important too]}


% ------------------------------------------------------------------------------
% Lattice QCD
\section{LQCD determinations of the nucleon quasi-elastic form factor\label{sec:lqcd}}
Lattice QCD is a discretized version of QCD, formulated in Euclidean spacetime, in which the quark fields live at the sites of the lattice and the gluon fields live on the links between the sites.  These link fields are given by Wilson-lines
\begin{equation}
U_\mu(x) = \exp\left\{i a\int_0^1 dt A_\mu(x +(1-t)a\hat{\mu}) \right\}
    \approx \exp\left\{i a \bar{A}_\mu(x) \right\}\, ,
\end{equation}
where $A_\mu(x)$ is the gluon field, $a$ is the ``lattice spacing'' and $\bar{A}_\mu(x)$ is the average value of $A_\mu(x)$ over the spacetime interval $[x, x+a\hat{\mu}]$.
This parameterization of the gauge fields allows for the construction of a discretized theory which preserves gauge-invariance~\addcite{Wilson}, a key property of gauge theories.
For example, the discretized Dirac operator
\begin{equation}
\bar{\psi}(x)\g_\mu D_\mu \psi(x) \rightarrow
\bar{\psi}(x)\g_\mu\frac{1}{2a}\left[U_\mu(x)\psi(x+a\hat{\mu}) -U^\dagger_\mu(x)\psi(x-a\hat{\mu}) \right]\, ,
\end{equation}
is invariant under gauge transformations,
\begin{align}
&\psi(x)\rightarrow \Omega(x)\psi(x)\, ,&
&U_\mu(x)\rightarrow \Omega(x)U_\mu(x)\Omega^{-1}(x+a\hat{\mu})\, .&
\end{align}


\begin{equation}
\hspace{-1.25in}\Umunu \hspace{-0.65in}
    =U_{\mu\nu}(x)
    =U_\mu(x)U_\nu(x+a\hat{\mu}) U^\dagger_\mu(x+a\hat{\nu}) U^\dagger_\nu(x)
\end{equation}



\bigskip
\begin{itemize}

\item The choice of Euclidean space is to allow Monte Carlo

\item comment on fermion and determinant

\end{itemize}



% ------------------------------------------------------------------------------
% LQCD Intro
%\subsection{OLD: Lattice QCD\label{sec:lqcd_intro}}
%In this section, we survey the present status of LQCD calculations of the nucleon axial form factor after providing a high-level introduction to LQCD.

%Lattice QCD is a discretized version of QCD, formulated in Euclidean spacetime, in which the quark fields live at the sites of the lattice and the gluon fields live on the links between the sites.  These link fields are given by Wilson-lines
\begin{equation}
U_\mu(x) = \exp\left\{i a\int_0^1 dt A_\mu(x +(1-t)a\hat{\mu}) \right\}
    \approx \exp\left\{i a \bar{A}_\mu(x) \right\}\, ,
\end{equation}
where $A_\mu(x)$ is the gluon field, $a$ is the ``lattice spacing'' and $\bar{A}_\mu(x)$ is the average value of $A_\mu(x)$ over the spacetime interval $[x, x+a\hat{\mu}]$.
This parameterization of the gauge fields allows for the construction of a discretized theory which preserves gauge-invariance~\cite{Wilson:1974sk}, a key property of gauge theories.
For example, the discretized Dirac operator
\begin{equation}\label{eq:naive_fermions}
\bar{\psi}(x)\g_\mu D_\mu \psi(x) \rightarrow
\bar{\psi}(x)\g_\mu\frac{1}{2a}\left[U_\mu(x)\psi(x+a\hat{\mu}) -U^\dagger_\mu(x)\psi(x-a\hat{\mu}) \right]\, ,
\end{equation}
is invariant under gauge transformations,
\begin{align}
&\psi(x)\rightarrow \Omega(x)\psi(x)\, ,&
&U_\mu(x)\rightarrow \Omega(x)U_\mu(x)\Omega^{-1}(x+a\hat{\mu})\, .&
\end{align}
Of note, the transformation of the link field maintains gauge invariance for the combindation of the $\bar{\psi}(x)$ and $\psi(x\pm a\hat{\mu})$ fields.


In the continuum, the gluon action-density is given by the product of field strength tensors, which are gauge-covariant curl's of the gauge potential
\begin{align}
&\mathcal{L}_G = \frac{1}{2g^2}\textrm{Tr}\left[G_{\mu\nu} G_{\mu\nu}\right]\, &
&G_{\mu\nu} = \partial_\mu A_\nu - \partial_\nu A_\mu +i [A_\mu, A_\nu]\, ,&
\end{align}
where $g$ is the gauge coupling.
When constructing the discretized gluon-action, it is therefore natural to use objects which encode this curl of the gauge potential.  The simplest such object is referred to as a ``plaquette'' and given by
\begin{equation}
\hspace{-1.25in}\Umunu \hspace{-0.65in}
    =U_{\mu\nu}(x)
    =U_\mu(x)U_\nu(x+a\hat{\mu}) U^\dagger_\mu(x+a\hat{\nu}) U^\dagger_\nu(x)\, .
\end{equation}
One can then show that the Wilson gauge-action reduces to the continuum action plus irrelevant (higher dimensional) operators which vanish in the continuum limit
\begin{align}\label{eq:gluon_action}
S_G(U) &= \beta \sum_{n=x/a} \sum_{\mu<\nu}
    \textrm{Re}\left[ 1 - \frac{1}{N_c} \textrm{Tr} \left[U_{\mu\nu}(n) \right]\right]
\nonumber\\&=
    \frac{\beta}{2N_c} a^4 \sum_{n=x/a,\mu,\nu} \frac{1}{2}
    \textrm{Tr} \left[ G_{\mu\nu}(n)G_{\mu\nu}(n)\right]
    +\mathrm{O}(a^6)\, ,
    & \rightarrow \beta = \frac{2N_c}{g^2}\, .
\end{align}
The continuum limit, which is the assymptotically large $Q^2$ region, is therefore approached as $\beta\rightarrow\infty$ where $g(Q^2)\rightarrow 0$.

There are many choices one can make in constructing the discretized lattice action.
Provided continuum QCD is recovered as $a\rightarrow0$, each choice is valid.
This is known as the universality of the continuum limit, with each choice only varying at finite lattice spacing.
Deviations from QCD, which arise at finite $a$, are often called \textit{discretization corrections} or \textit{scaling violations}.
That all lattice actions reduce to QCD as $a\rightarrow0$ is known as the universality of the continuum limit.  It is a property which can be proved in perturbation theory but must be established numerically given the non-perturbative nature of QCD.
For sufficiently small lattice spacings, one can use effective field theory (EFT) to construct a continuum theory that encodes the discretization effects in a tower of higher dimensional operators.  This is known as the Symanzik EFT for lattice actions~\cite{Symanzik:1983dc,Symanzik:1983gh}.
One interesting example involves the violation of Lorentz Symmetry at finite lattice spacing: in the Symanzik EFT, the operators which encode this Lorentz violation scale as $a^2$ with respect to the operators which survive the continuum limit, and thus, Lorentz symmetry is an accidental symmetry of the continuum limit.  It is not respected at any finite lattice spacing, but the measurable consequences vanish as $a^2$ for sufficiently small lattice spacing.


The inclusion of quark fields adds more variety of lattice actions.
One main complication for fermions is that the ``naive discretization'', equation~\eqref{eq:naive_fermions}, leads to a well-known doubling problem in which, instead of a single fermion, one has $2^d$ doubler fermions, or poles in the propagator in $d$ dimensions.
In particular, it is challenging to remove these doublers without breaking chiral symmetry with the lattice regulator, an issue known as the Nielsen-Ninomiya No Go Theorem~\cite{Nielsen:1981hk,Nielsen:1980rz,Nielsen:1981xu}.
There are four commonly used fermion discretization schemes that deal with this no-go theorem in different ways:
\begin{itemize}[leftmargin=*]
\item Staggered, or Kogut-Susskind fermions~\addcite{KS}, split the four components of the fermion spinor onto different corners of a local hyper-cube, reducing the doublers by a factor of four.  A fourth root of the fermion determinant is then used to simulate a single fermion flavor.  Staggered fermions retain a remnant chiral symmetry, protecting the quark mass from additive mass renormalization.  They are the least expensive numerically to simulate, and thus state of the art computations now include six lattice spacings to control the continuum extrapolation.  Splitting the spinor components to different lattice sites complicates the Dirac structure of the theory, mixing it with the spacetime symmetries, and complicating the construction of hadrons with spin, such as nucleons~\addcite{Aaron knows some refs}.

\item Clover-Wilson fermions add two irrelevant dimension-5 operators to the action.  The first is a double derivative operator (the Wilson opeator) that gives a mass to the doublers that scales as $1/a$, thus lifting all but one of the doublers to be a heavy state that decouples in the continuum limit.  The double derivative operator explicitly breaks chiral symmetry causing an $1/a$ additive shift to the quark mass, thus requiring fine-tuning to retain the small up and down quark masses.
The second operator is the clover operator, $a c_{SW} \bar{q} \sigma_{\mu\nu} G_{\mu\nu} q$.  The $c_{SW}$ coefficient~\addcite{Csw} can be tuned to remove the residual $\mathrm{O}(a)$ chiral symmetry breaking discretization effects from the spectrum~\addcite{O(a) papers}.  Removing such $\mathrm{O}(a)$ effects from all quantities, such as nucleon matrix elements and form factors, requires the tuning of additional irrelevant operators to remove such scaling violations from currents and other operators.  This procedure is well understood, and routinely carried out, but is also laborious and time consuming.  Nevertheless, since Wilson type fermions respect all other symmetries and they are relatively inexpensive to simulate, they are the most widely used in the literature.

\item Twisted-mass fermions~\addcite{tm papers} are a variant of Wilson or clover-Wilson fermions where the approximate $SU(2)$ flavor symmetry of the light quarks is used to add a mass term proportional to $i\gamma_5 \tau_3$.  This mass terms is protected from additive mass renormalization, and so if one works at \textit{maximal twist}, such that the regular mass term just cancels the $1/a$ mass term from the Wilson operator, the theory is automatically $\mathrm{O}(a)$ improved, meaning the leading $\mathrm{O}(a)$ chiral symmetry breaking corrections that normally arise for Wilson fermions are removed.
Twisted mass fermions break isospin symmetry at finite lattice spacing, causing some complications now that LQCD results are precise enough to require isospin breaking corrections from $m_d-m_u$ and QED to be compared with experiment.

\item The fourth most common discretization are Domain Wall Fermions (DWF)~\addcite{}, which introduce a fifth dimension to the theory with unit links (the gluons are not dynamic in the fifth dimension).  The left and right handed fermions are bound to opposite sides of the fifth dimension.  The overlap of these left and right modes gives rise to an explicit chiral symmetry breaking that is exponentially suppressed by the extent of the fifth dimension.  For sufficiently small chiral symmetry breaking (large $L_5$), DWF are also automatically $\mathrm{O}(a)$ improved.
While very desirable, DWF are numerically more expensive to simulate, both because of the extra fifth dimension and also because the algorithmic speed up offered by multi-grid, which works tremendously for clover-Wilson fermions~\addcite{Kate + Balint Titan/Summit}, is not yet flushed out for DWF~\addcite{Boyle and QUDA people}.

\item A final common variant of action is one in which the fermion discretization used in the generation of the gauge fields (the sea quarks) and the action used when generating quark propagators (the valence quarks) are different: this is known as a \textit{mixed action}~\cite{Renner:2004ck}.
The most common reason to use such an action is to take advantage of numerically less expensive methods to generate the configurations while retaining good chiral symmetry properties of the valence quarks, which is known to suppress chiral symmetry breaking effects from the sea-quarks~\cite{Bar:2002nr,Bar:2005tu,Tiburzi:2005is,Chen:2007ug}.


\end{itemize}
There is an extensive literature on LQCD and its application to computing properties of nucleons.
For reviews concerning nucleon structure, see Refs.~\addcite{}.
For in depth introductions to lattice QCD, see the text books~\cite{Smit:2002ug,DeGrand:2006zz,Gattringer:2010zz}.






\addcomment{thoughts for rest of intro to LQCD}
\begin{itemize}

\item refer to reviews of LQCD and text books, Smitt, Gatringer and Lang, DeTar and DeGrand

\item full calculation requires extrapolation to the continuum limit, physical pion mass, infinite volume

\item 2pt function

\item {\color{red}[move to next section?]} extra challenges for FF: light pions = more noise, (light pions more expensive cause larger L, larger condition number, and more statistics needed); finite L limits accessible kinematic points - particularly challenging for extracting slope of form factors at small $Q^2$ - FF parameterizations can show non-trivial structure in between kinematically accessible points - leads to ideas to compute slope of FF as well as FF...

\end{itemize}


% ------------------------------------------------------------------------------
% anatomy of LQCD calculation
\subsection{Anatomy of LQCD Calculations of Nucleon Form Factors\label{sec:calc_anatomy}}
In order to determine the mass of the nucleon (or any hadron), we rely on the construction of two-point correlation functions that are constructed in a mixed time-momentum representation, most commonly utilizing spatially local creation operators (sources) and momentum space annihilation operators (sinks).
The non-perturbative nature of QCD means we do not know how to construct the nucleon wave function, and so we utilize \textit{interpolating operators} which have the quantum numbers of the state we are interested in.
These creation and annihilation operators will couple to all eigenstates of QCD with the same quantum numbers giving rise to a two-point function with a spectral decomposition
\begin{align}\label{eq:2pt}
    C(t,\mathbf{p}) &= \sum_{\mathbf{x}} e^{-i \mathbf{p\dotp x}}
        \langle \O| O(t,\mathbf{x}) O^\dagger(0,\mathbf{0}) | \O \rangle
%\nonumber\\&=
    =
    \sum_{n=0}^\infty z_n(\mathbf{p}) z_n^\dagger(\mathbf{p}) e^{-E_n(\mathbf{p})t}\, .
\end{align}
In this expression, $|\O\rangle$ is the vacuum state,
$z_n(\mathbf{p}) = \sum_{\mathbf{x}}e^{-i\mathbf{p\dotp x}} \langle \O|O(0,\mathbf{x})|n\rangle$
and $z_n^\dagger(\mathbf{p}) = \langle n(\mathbf{p})|O^\dagger(0,\mathbf{0})|\O\rangle$.
To go from the first equality to the second, we have inserted a complete set of states, $1=\sum_n |n\rangle\langle n|$ and we have used the time-evolution operator to shift the annihilation operator to $t=0$ and expose the explicit time dependence.%
%-------------------------------------------------------------------------------
\begin{marginnote}
\entry{$O(t,\mathbf{x}) = e^{\hat{H}t} O(0,\mathbf{x}) e^{-\hat{H}t}$}{The Hamiltonian, $\hat{H}$, is used to time evolve the operator}
\end{marginnote}%
%-------------------------------------------------------------------------------
Momentum conservation selects only the state $n(\mathbf{p})$ in the sum over all states enabling the use of this mixed spatial creation operator and momentum annihilation operator.
As we will discuss in more detail below, it is more desirable to instead use momentum space creation operators.
They are not commonly used as they are significantly more expensive numerically to generate.

For large Euclidean time, the correlation function will be dominated by the ground state as the excited states will be exponentially suppressed by the energy gap
\begin{equation}
    C(t) = z_0 z_0^\dagger e^{-E_0 t}\left[
        1 + r_1 r^\dagger_1 e^{-\Delta_{1,0}t} + \cdots \right]\, .
\end{equation}%
\begin{marginnote}
\entry{$\D_{m,n}= E_m - E_n$}{}
\entry{$r_n = z_n / z_0$}{}
\end{marginnote}%
It is useful to construct an \textit{effective mass} to visualize at which the ground state begins to saturate the correlation function
\begin{align}
m_{\rm eff}(t) &= \ln \left( \frac{C(t)}{C(t+1)} \right)
%\nonumber\\&=
    =
    E_0 + \ln\left( 1 + \sum_{n=1} r_n r^\dagger_n e^{-\Delta_{n,0}t}\right)\, .
\end{align}


A well known challenge which complicates the analysis is that for nucleon two-point functions, the S/N ratio degrades exponentially at large Euclidean time~\cite{Lepage:1989hd}
\begin{equation}
\lim_{{\rm large}\ t}\frac{\mathrm{Signal}}{\mathrm{Noise}}
    \propto \sqrt{N_{\mathrm{sample}}} e^{(m_{\mathrm{N}} - \frac{3}{2}m_\pi)t}\, ,
\end{equation}
which adds an extra source of excited state systematic uncertainty that must be quantified:
the region in time when the ground state begins to saturate the correlation functions, at $t\approx 1$~fm, the noise is becoming significant, making the correlation functions in this region susceptible to correlated fluctuations which can bias a simplistic single-state analysis.
Further, as the pion mass is reduced towards its physical value, the excited state energy gap also shrinks, as the lowest lying excited state is typically a nucleon-pion in a relative $P$-wave.  At the same time, the energy scale which governs the exponential degradation of the signal also grows.  The former issue means calculations must be performed at larger Euclidean time to suppress the smaller gapped excited states and the latter issue means we need exponentially more statistics to obtain a fixed relative uncertainty at a given time.
In order to boost statistics, given the numerical cost of generating more configurations, time and spatial translation invariance are used to generate sources at several choices of $t_0$ and $\mathbf{x}_0$, which are then translated back to the origin and averaged together.  For the nucleon, it was observed that hundreds of sources on each configuration, using an anisotropic clover-Wilson ensemble~\cite{HadronSpectrum:2008xlg}, still showed approximate $\sqrt{N_{\rm sample}}$ improvement of the stochastic uncertainty as a function of the number of sources per configuration~\cite{Beane:2009kya}.



The most common method of constructing three-point correlation functions follows a similar strategy, beginning with spatially local sources.
A nucleon three-point function with current $j_\G$ is contstructed with interpolating operators $N(\tsep,\mathbf{x})$ and $N^\dagger(0,\mathbf{0})$,%
%-------------------------------------------------------------------------------
\begin{marginnote}
\entry{$j_\G = \bar{q}\, \G\, q$}{quark bilinear currents of Dirac structure $\G$ and unspecified flavor structure}
\end{marginnote}%
%-------------------------------------------------------------------------------
\begin{align}
C_\G(\tsep,\t) &= \sum_{\mathbf{x,y}}e^{-i\mathbf{p\dotp x} +i\mathbf{q\dotp y}}
    \langle\O|N(\tsep,\mathbf{x}) j_\G(\t,\mathbf{y}) N^\dagger(0,\mathbf{0}) |\O\rangle
\nonumber\\&=
    \sum_{n,m} z_n(\mathbf{p})z_m^\dagger(\mathbf{p-q})e^{-E_n(\tsep-\t)}e^{-E_m \t} g_{n,m}^\G\, ,
\end{align}
where $g_{n,m}^\G$ are the matrix elements of interest, and in principle, all other quantities can be determined from the two-point function, a point we will return to.%
%-------------------------------------------------------------------------------
\begin{marginnote}
\entry{$g_{n,m}^\G = \langle n| j_\G |m\rangle$}{hadronic matrix elements of interest from state $m$ to $n$ with implicit momentum and energy dependence}
\end{marginnote}%
%-------------------------------------------------------------------------------
Often, the sink is projected to zero momentum, $\mathbf{p}_n=0$, for which momentum conservation gives the momentum of the incoming state to be $\mathbf{p}_m = -\mathbf{q}$.
A typicall calculation is performed with a sequential propagator~\cite{Martinelli:1988rr}
whose source is constructed by taking the forward propagators from the origin and contracting all but one spin and color index at the sink time, $\tsep$.%
% FOOTNOTE ---------------------------------------------------------------------
\footnote{\color{red}Add comment about Adelaide FH method and one-end trick} 
%-------------------------------------------------------------------------------

Each choice of $\tsep$, flavor and spin of the initial and final state, and each choice of flavor for the current requires a new sequential propagator, rendering this aspect of the computation relatively expensive.
While each sequential propagator can only be used for these specific choices, one is free to insert any quark bilinear operator for the current, including non-trivial spatial structure and momentum.
To boost statistics, in addition to taking advantage of translation invariance, as with the two-point function, the \textit{coherent sink technique}~\cite{LHPC:2010jcs} is used to solve a single sequential propagator for many choices of the origin.

The most significant challenge in determining the nucleon matrix elements and subsequent form factors is dealing with the excited state contamination, an issue that is compounded by the degrading S/N.
If the nucleon two-point function is becoming saturated by the ground state at $t\approx1$~fm, the ideal three-point function would use values of $\tsep\approx2$~fm.
In practice, a few values of $\tsep$ in the range $0.8 \lesssim \tsep \lesssim 1.5$~fm are used because the S/N ratio of the three-point functions decays more rapidly than the two-point function, and an extrapolation to large $\tsep$ is used to isolate the ground state matrix elements.
However, using just one excited state in the analysis requires three values of $\tsep$ in order to have a one degree-of-freedom fit to extract the ground state matrix element.
Following Ref.~\cite{Chang:2018uxx}, that utilized many values of $\tsep$ to determine $g_A$, a few groups have begun advocating for the use of many values of $\tsep$, including small values, to improve our ability to understand and control the excited state contamination~\cite{Hasan:2019noy,Alexandrou:2019brg,He:2021yvm}.
Ref.~\cite{He:2021yvm} utilized 13 values of $\tsep$, which allows for a systematic study of the uncertainty associated with truncating early and/or late values of $\tsep$ in the analysis, as well as providing sufficient data to perform up to a complete 5-state fit.






\bigskip\noindent
{\color{red}[points to add]}
\begin{enumerate}

\item 2pt functions are not sufficient to constrain excited state spectrum,

\item 3pt functions do a much better job, therefore, fitting 2pt and feeding into 3pt underestimates uncertainty and may bias results.  Also, no ability to resolve ``nature'' of state (Npi or Npipi)

\item non-zero momentum means parity mixing, further complicating analysis (should also fit parity odd spec and include)

\item momentum smearing for higher momentum states, S goes with non-zero P but N goes with P=0, increasing the energy scale associated with S/N degradation;  for larger Qsq, use of momentum smeared sources

\item the energy of the in/out states are different, complicating the analysis and preventing the use of FH/summation method

\item excited state contamination seems to ruin PCAC (next section or merge with this one?)

\end{enumerate}


% ------------------------------------------------------------------------------
% LQCD PCAC
\subsection{Role of PCAC in LQCD results of $F_{\mathrm{A}}(Q^2)$\label{sec:lqcd_pcac}}
\cw{Some repetition between this paragraph and Section 2.}
Computations of the nucleon axial charge have long been considered
 a vital benchmark for nucleon physics with lattice QCD.
This quantity has been precisely measured with experiments probing neutron beta decay,
 characterized by a weak decay with a low momentum transfer.
Historically, lattice calculations of the axial charge have obtained values
 that were $O(10\%)$ too low, despite significant investments of effort.
This has been the subject of some controversy,
 with more and more sophisticated calculations to scrutinize lattice systematics.
Over time, contamination from excited states was identified as
 an important contributing factor to the systematic deviation.
Many collaborations have opted for high-statistics computations
 to reduce the statistical uncertainty at late times and permit
 multi-exponential fits to the time dependence of the correlation functions.
As time progressed, LQCD results moved closer to the experimental value,
 and now modern calculations are in agreement with experiment
 at the 1\% level~\cite{Kronfeld:2019nfb}.
\textcolor{red}{[more citations]}
Modern efforts to compute nucleon matrix elements now target the nucleon
 axial form factor with its four-momentum transfer dependence.

To check the accuracy of lattice QCD calculations targeting the axial form factor,
 the validity of the partially-conserved axial current (PCAC) relation,
\begin{align}
 \partial^\mu A^{a}_{\mu}(x) = 2 m_q P^{a}(x),
 \label{eq:pcac}
\end{align}
 has been scrutinized with lattice data.
The PCAC relation is an exact symmetry in the continuum limit.
The generalization of this relation to the nucleon form factors
 yields the generalized Goldberger-Triemann (GGT) relation,
\begin{align}
 2 M_N G_A(Q^2) -\frac{Q^2}{2M_N} \widetilde{G}_P(Q^2) = 2 m_q G_{P}(Q^2),
 \label{eq:ggt}
\end{align}
 which provides orthogonal checks of individual matrix elements
 for the axial and pseudoscalar currents.
The pion pole dominance (PPD) ansatz is also studied,
\begin{align}
 \widetilde{G}^{\rm PPD}_P(Q^2) = \frac{4M_N^2}{Q^2+M_\pi^2} G_A(Q^2),
 \label{eq:ppd}
\end{align}
 which is only approximate even in the continuum and is obtained
 by carefully considering the leading asymptotic behavior of the
 form factors in the double limit $Q^2\to0$ and $m_q\to0$~\cite{Sasaki:2007gw}.

Initial calculations targeting the axial form factor verified the PCAC relation
 for the full correlation functions but found significant \emph{apparent} violations
 of the Generalized Goldberger-Triemann relation, Eq.~\ref{eq:ggt}.
The resolution of the apparent violation of GGT
 is now informed by baryon chiral perturbation theory, which suggests that chiral
 and excited state corrections to the spatial axial, temporal axial, and induced pseudoscalar
 are functionally different and not properly removed.
The axial contributions are largely dominated by $N\pi$ excited states
 with a highly suppressed single nucleon chiral correction.
The correction to the axial current is nearly independent of $Q^2$.
On the other hand, corrections to the induced pseudoscalar are
 driven by the single nucleon chiral contribution and has
 a strong $Q^2$ dependence~\cite{Bar:2018xyi}, with the largest correction at low $Q^2$.
The $N\pi$ contribution in the induced pseudoscalar is highly suppressed by
 an approximate cancellation.
The contamination to the pseudoscalar current is redundant with the
 axial and induced pseudoscalar chiral corrections and can be obtained
 by application of the PCAC relation.

Scrutiny of the lattice QCD data has demonstrated many of the features
 that were expected from chiral perturbation theory.
The primary excited state contaminations to the axial matrix elements
 were shown to be driven by two specific $N\pi$ states,
 characterized by a transition through an axial current
 of the nucleon state to an $N\pi$ excited state or vise versa~\cite{Jang:2019vkm}.
\textcolor{red}{[is below statement understandable?]}
The momenta of the states with the strongest contribution are those
 where a valence quark diagrams can be drawn containing only a single
 zero-momentum gluon emission, which fixes the relative momenta of the
 quark constituents making up the nucleons and the pion.
These $N\pi$ states were initially expected to be negligible due to a volume suppression
 of the state overlap, which makes them invisible to the two-point functions~\cite{Bar:2016uoj}.
However, the three-point axial matrix element enhances these contributions relative
 to the ground state nucleon matrix element, which is enough to overcome the volume suppression.
As a consequence, analyses that fix the spectrum using the two-point functions alone
 will often miss the important $N\pi$ contamination to the
 axial matrix element~\cite{Jang:2019vkm,He:2021yvm}.

The lattice data also showed deviations as large as $40\%$ from the pion pole
 dominance ansatz at low $Q^2$, where it was expected to
 work best~\cite{Bali:2014nma,Gupta:2017dwj}.
Each $Q^2$ prefered dominant $N\pi$ states with different energies,
 which when neglected across the full range of momentum transfers
 produced a $Q^2$-dependent discrepancy in excess of the $Q^2$ behavior
 expected from the GGT and PPD relations.
Fits to the three-point functions that allow for the possibility of
 nonnegligible $N\pi$ states are able to constrain the effects of these states,
 removing the contamination and thereby restoring the PPD relation.
\textcolor{red}{[ could fit to temporal axial too ]}



% ------------------------------------------------------------------------------
% LQCD results
\subsection{Survey of LQCD results of $F_{\mathrm{A}}(Q^2)$\label{sec:lqcd_results}}

\textcolor{red}{[NME]}
\textcolor{red}{[RQCD]}

\textcolor{red}{[ETMC]}
The ETMC calculation~\cite{Alexandrou:2020okk} of the axial form factor
 is performed on three ensembles with twisted mass fermions all at physical pion mass.
Two of these ensembles include only light quarks in the sea,
 leading to unquantifiable systematic corrections from
 neglecting the effects of strange quarks.
However, these two two-flavor ensembles permit an explicit test of finite-volume corrections,
 demonstrating a lack of dependence on the volume in the range of typical lattice computations.
The remaining ensemble has four flavors of sea quarks and thus is not subject
 to the same criticism of neglecting the strange quarks.
Despite the agreement with the trend of axial form factor results from other collaborations,
 the results from this computation do not satisfy the PPD and GGT relations,
 suggesting remnant excited state contamination.

\textcolor{red}{[PACS]}
The PACS collaboration has focused on computing the axial form factor
 on large ensembles with Wilson clover fermions at
 physical pion mass~\cite{Ishikawa:2018rew,Shintani:2018ozy}.
The ensemble used in this analysis has a volume of $(10.8~{\rm fm})^3$,
 considerably larger than ensembles used by other collaborations.
The large volume reduces the minimum $Q^2$ that can be probed
 with the discrete lattice momenta but forces the need for many units of momenta
 to access the same kinematic range as other computations.
They compare this ensemble to a computation of the axial radius using
 the traditional method compared with derivatives obtained from
 moments of the correlators~\cite{Aglietti:1994nx},
 on a $(5.5~{\rm fm})^3$ volume ensemble, finding agreement between the results of each.
Results on both ensembles are about $1\sigma$ small compared to the axial radius
 from the average from experimental sources in Ref.~\cite{Hill:2017wgb}.
Like the ETMC results, the PACS calculation does not satisfy the PPD and GGT relations.

\textcolor{red}{[CLS]}
The CLS collaboration have an axial form factor computation on a
 single ensemble of Wilson clover fermions at physical pion mass~\cite{Hasan:2017wwt}.
This methodology paper focuses on computing the nucleon charges
 and radii by explicitly computing derivatives of the correlation functions
 with respect to the momentum transfer.
This method yields the axial radius directly from $Q^2=0$ data,
 which is compared to the nucleon form factor slopes and charges obtained from
 fits to data at nonzero $Q^2$ using the traditional three-point correlator methods.
Though the results are in agreement between the two methods,
 the direct derivative proves to be noisier than the traditional method.
With more precision, the axial radius obtained from this method,
 or constraints on the slope obtained in a similar fashion at nonzero $Q^2$,
 could provide orthogonal constraints on the form factor that could
 help pin down the axial form factor shape.

\textcolor{red}{[Other calculations]}
In addition to the aforementioned published results,
 a handful of recent preliminary results that deserve mention.
The Mainz collaboration has an ongoing calculation on 12 ensembles,
 including an ensemble at physical pion mass
 and a chiral-continuum and infinite volume extrapolation~\cite{Djukanovic:2021yqg}.
This computation is performed with improved Wilson fermions.
The Fermilab Lattice and MILC collaborations also have an ongoing
 computation of the axial form factor using a unitary HISQ-on-HISQ setup,
 for which a preliminary computation of the axial charge on
 a single unphysical ensemble exists~\cite{Lin:2020wko}.
Because of the choice of action, this computation has more nucleon ``tastes''
 than other efforts, which is more computationally affordable
 at the cost of a more challenging analysis.
The CalLat collaboration have a computation of the axial form factor
 using a mixed-action domain wall on HISQ setup,
 with existing data on several ensembles including multiple physical pion mass ensembles.
The fits to one physical mass ensemble from the CalLat collaboration are used
 to make statements about the pheno impact of the LQCD results in Sec.~\ref{sec:impact}.

\begin{figure}[hbt!]
\centering
\includegraphics[width=0.5\textwidth]{plots/gaq2-overlay-standalone.pdf}
\caption{
Published results for the axial form factor obtained from lattice QCD,
 compared with the deuterium extraction from Ref.~\cite{Meyer:2016oeg}.
Collaborations that have obtained their results from only a single ensemble
 are plotted as scatter points.
These single-ensemble results will have small but unknown corrections due to chiral, continuum,
 and finite volume systematic shifts.
The NME~\cite{Park:2021ypf} and RQCD~\cite{RQCD:2019jai}
 results are both obtained from fits to several ensembles.
The RQCD perform the full chiral-continuum and finite volume extrapolations to the data,
 fitting to each of the form factors independently for each ensemble but providing
 the constraint that the form factors must satisfy the GGT relation in the continuum.
The NME collaboration also performed a chiral-continuum and finite volume extrapolation
 on their data, but found significantly larger uncertainties and so their result
 is obtained from an average of the results on their five largest volume ensembles.
As a rough estimate, NME claims that the uncertainties are at least a factor of 5 larger
 if the fits are relaxed to allow the results to change with pion mass or lattice spacing.
}
\end{figure}

\begin{itemize}
\item
\textcolor{red}{[
 temporal axial current - subject to large excited state corrections.
 ETMC claim precise, use it to constrain $N\pi$.
]}
\item
\textcolor{red}{[
 Axial form factor has different corrections than induced pseudoscalar.
 Most of $N\pi$ contamination introduced into GGT and PCAC from induced pseudoscalar.
]}
\item
\textcolor{red}{[ ensemble details of calculations.]}
\item
\textcolor{red}{[
 $g_A(Q^2)$ parameterizations.
 $z$ expansion intro, taken from chiral section.
]}
\item
\textcolor{red}{[
 Different methods of dealing with excited states.
 FH, 3pt fits, summation, other?
]}
\item
\textcolor{red}{[
 axial form factor with full set of coefficients and covariance needed
 in chiral-continuum-FV limit.
 $r_A^2$ reported to connect with pion electroproduction/neutron decay/very low-$Q^2$ applications,
 but not nearly as important for neutrinos.
]}
\item
\textcolor{red}{[ Need an axial ff calculation with explicit $N\pi$-like operators. ]}
\item
\textcolor{red}{[ Will need to verify systematics for large $Q^2$ axial ff,
 but chiral corrections/excited state contaminations are also smallest in this region.
 May be fooled into comfort by breakdown of XPT.
]}
\item
\item
\textcolor{red}{[
 failure of dipole parameterization with existing LQCD data.
 consistency with expt for $r_A^2$ but high at large $Q^2$ means
 that one-parameter fits are insufficient to describe $Q^2$ behavior.
 ]}
\end{itemize}

Citations for $F_A(Q^2)$ references pulled from NME21, no $g_A$-only references:
\begin{description}
\item[NME 21]~\cite{Park:2021ypf}
\item[RQCD 20]~\cite{Bali:2018qus,RQCD:2019jai} %% 63
\item[ETMC 20]~\cite{Alexandrou:2018sjm,Alexandrou:2019brg,Alexandrou:2020okk} %% 54-56
\item[PACS 18 (erratum)]~\cite{Ishikawa:2018rew,Shintani:2018ozy} %% 60-61 (62 proceedings)
\item[PNDME 17]~\cite{Gupta:2017dwj,Gupta:2018qil,Jang:2019vkm,Jang:2019jkn} %% 6-9
\item[CLS 17]~\cite{Hasan:2017wwt,Hasan:2019noy} %% LHPC? 66-67
\end{description}
Other axial ff effort references:
\begin{description}
\item[new PACS gA(Q2)]~\cite{Ishikawa:2021eut}
\item[Mainz gA(Q2)]~\cite{Djukanovic:2021yqg}
\item[Fermilab Lattice+MILC HISQ gA(0)]~\cite{Lin:2020wko}
\item[CalLat gA(Q2)]~\cite{Meyer:2021vfq}
\end{description}
Other references:
\begin{description}
\item[USQCD white paper]~\cite{Kronfeld:2019nfb}
\item[FLAG 21]~\cite{Aoki:2021kgd}
\item[CalLat excited states $g_A$]~\cite{He:2021yvm}
\item[Ottnad excited states]~\cite{Ottnad:2020qbw}
\item[Baer XPT]~\cite{Bar:2018xyi,Bar:2019igf}
\item[Axial radius from LQCD using XPT]~\cite{Yao:2017fym}
\end{description}



% ------------------------------------------------------------------------------
% z-expansion
\subsection{Combining the $z$-expansion with the continuum and chiral extrapolations\label{sec:z_continuum}}

Lattice data are computed with nonzero lattice spacings and typically at unphysical pion masses.
To control for the effects of these systematics,
 chiral perturbation theory and a Symanzik effective theory are typically employed.
These connect the lattice results to the physical point by means
 of a perturbative expansion in parameters such as the pion mass or lattice spacing.
Powers of these expansion parameters come with some low energy constants (LECs)
 that must be fit to the lattice data over various ensembles with different parameters.
With the LECs in hand, extrapolation to the continuum physical point
 may be completed and results obtained for QCD.

These extrapolations are known for the axial charge,
 but chiral expansions in $Q^2$ have not been formulated as of yet.
In their place, one might choose to instead parameterize the form factor as
 a Taylor expansion in $Q^2$,
\begin{align}
 F(Q^2) = \sum_{k=0} \ell_k (Q^2)^k,
\end{align}
 where the LECs $\ell_k$ have dependence on the pion mass and lattice spacing.
This form factor expansion only has limited validity at $Q^2$ close to zero,
 failing as higher orders become important.

The $z$ expansion is a conformal mapping from the four-momentum transfer squared $Q^2$
 to a small expansion parameter $z$, parameterized by the relation
\begin{align}
 z(t=-Q^2;t_0,t_c) = \frac{\sqrt{t_c-t} -\sqrt{t_c-t_0}}{ \sqrt{t_c-t} +\sqrt{t_c-t_0}},
\end{align}
 where $t_c$ is the kinematic cutoff in timelike momentum transfer for
 a particle production threshold and $t_0$ is a parameter (typically negative)
 that may be chosen to improve the series convergence.
Inverting this relation and expanding as a power series in $z$ about $Q^2=-t_0$ yields
\begin{align}
 Q^2+t_0 = 4 (t_c-t_0) \sum_{k=1}^\infty k z^k.
 \label{eq:Q2toz}
\end{align}
A general series in $Q^2$ may therefore be converted to an expansion in $z$ and $t_0$
 by first converting powers of $Q^2$ to those of $Q^2+t_0$ and $t_0$ using binomial theorem,
\begin{align}
 Q^{2m} &= \big( (Q^2+t_0) -t_0 \big)^m
 = \sum_{n=0}^{m} \left( \begin{array}{c} m \\ n \end{array} \right) (Q^2+t_0)^n (-t_0)^{m-n}
\end{align}
 and then using Eq.~(\ref{eq:Q2toz}) to convert powers of $Q^2+t_0$ into a power series in $z$.
Following this procedure, the LECs that appear as prefactors for powers of $Q^2$
 are instead assembled into expressions related to the coefficients of the $z$ expansion,
\begin{align}
 F\big(z(Q^2)\big) = \sum_k a_k z^k.
 \label{eq:zexp}
\end{align}
To obtain an expansion to order $O\big(z^{k}\big)$,
 a starting expansion of order $O\big((Q^2)^{k}\big)$ is sufficient.

When transforming the expansion in $Q^2$ into an expansion in $z$,
 the one expansion parameter $Q^2$ is replaced instead by two new parameters, $z$ and $t_0$.
The parameter $t_0$ is arbitrary and can be chosen to be small (or 0) based on preference.
The relative size of $t_0$ and $z$ are inversely related,
 so by making $t_0$ larger a higher expansion order in $t_0$
 may be traded for a lower expansion order in $z$.
However, $t_0$ must still be kept small enough that the expansions in $t_0$ converge.

For the inverse transformation to take an expansion in $z$ to an expansion in $Q^2$,
 the second expansion parameter that appears is $\sqrt{t_c/(t_c-t_0)}$ instead of $t_0$.
To demonstrate this, take $\alpha=\sqrt{(t_c-t)/(t_c-t_0)}$ and expand
\begin{align}
 z = \frac{\alpha-1}{\alpha+1} = 1 - 2 \sum_k (-\alpha)^k,
\end{align}
 then expanding $\alpha^k$ in $t$ yields
\begin{align}
 \alpha^k &= \biggr(\frac{ t_c}{ t_c -t_0} \biggr)^{k/2} \biggr( 1 -\frac{t}{t_c} \biggr)^{k/2}
 = \sum_{n=0} \biggr(\begin{array}{c} k \\ n \end{array} \biggr)
 \biggr(\frac{ t_c}{ t_c -t_0} \biggr)^{k/2} \biggr( -\frac{t}{2t_c} \biggr)^{n}.
\end{align}
This transformation instead prefers \emph{larger} $-t_0$ such that
 $\sqrt{t_c/(t_c-t_0)}<1$ and does not converge for $t_0=0$.
The term that appears with $n=0$ is a $t$-independent term that is multiplied by
 the unknown LEC $a_k$ in the $z$ expansion of Eq.~(\ref{eq:zexp}).

The power series in $z$ fixes the $Q^2$ dependence of the form factor
 by building in the correlations required by analyticity
 between LECs of measurable low powers of $Q^2$
 and in practice unmeasurable higher powers.
Further constraints from perturbative QCD may be used to bound the size of higher-order LECs
 in the $z$ expansion.




% ------------------------------------------------------------------------------
% Impact
\section{Phenomenological Impact\label{sec:impact}}
Neutrino oscillation experiments measure an event rate, which is the convolution of the flux, cross section and detector efficiency, as a function of some measureable variable. The incoming neutrino energy is not known event by event, and not all outgoing particles are detectable, so quantities such as the energy transfer, or four-momentum transfer, cannot be reconstructed. As neutrino oscillation is a neutrino energy (and distance) dependent phenomenon, experiments attempt to reconstruct it using the kinematics of particles produced when neutrinos interact in their detectors.

T2K, and other experiments with a relatively low energy ($\lessapprox1$ GeV) beam, attempt to reconstruct the neutrino energy using outgoing lepton momentum, $p_{l}$, and its angle with respect to the incoming beam direction, $\theta_{l}$, assuming two-body quasi-elastic kinematics with the initial nucleon at rest,
\begin{equation}
E^{\mathrm{rec,\;QE}}_{\nu}\left(p_{l}, \theta_{l}\right) = \frac{2m_f\sqrt{p_{l}^2 + m^2_l} - m_l^2 + m_i^2-m_f^{2}}{2\left(m_f-\sqrt{p_{l}^2 + m^2_l}+p_l \cos\theta_l\right)},
\label{eq:enuqe}
\end{equation}
\noindent where $m_l$ is the mass of the outgoing lepton, $m_{i}$ is the mass of the initial state nucleon, and $m_{f}$ is the mass of the final state nucleon. As this variable assumes quasi-elastic kinematics, it is applied to a signal sample of events with a muon, no pions or other mesons, and any number of nucleons produced in the final state~\footnote{Note that in recent analyses, T2K has included samples including pions using a modified version of Equation~\ref{eq:enuqe}~\addcite.} (CC0$\pi$). Events that are not true CCQE events also contribute to the CC0$\pi$ signal, such as charged-current interactions with two nucleons (CC-2p2h) or charged-current interactions with resonant production but no visible final state pion (CC-RES). The two-body approximation in Equation~\ref{eq:enuqe} is a very poor approximation of the true neutrino energy, $E_{\nu}^{\mathrm{true}}$, in these cases. Understanding the relative fraction of the different interaction channels is therefore a critical issue for experiments that use Equation~\ref{eq:enuqe}.

\begin{figure}[htbp]
  \centering
  \captionsetup[subfloat]{captionskip=-5pt}
  \subfloat[Near detector]{\includegraphics[width=0.3\textwidth]{plots/T2KND_numu_H2O_model_comp.pdf}}\hspace{75pt}
  \subfloat[Far detector] {\includegraphics[width=0.3\textwidth]{plots/T2KFD_numu_H2O_osc_model_comp.pdf}}
  \vspace{11pt}
  \caption{The $\nu_{\mu}$--H$_{2}$O CC0$\pi$ event rates per ton (kiloton) per $1\times10^{21}$POT at T2K's near (far) detector site, shown as a function of $E^{\mathrm{rec,\;QE}}_{\nu}$. The GENIE nominal event rate (blue solid line) is produced using the GENIEv3 10a\_02\_11a tune to nucleon data~\addcite{} and the T2K flux~\addcite{}, and the CCQE (orange dashed line), CC-2p2h (red short dashed line) and CC-other (green dotted line) contributions are shown. The oscillated flux is calculated with Ref.~\addcite{}, using the best fit NuFit5.0 oscillation parameters in normal ordering. Additionally, an alternative GENIE model is shown, where the only change is to use the z-expansion model of the axial form factor, with parameters tuned to LQCD results, as described in Section~XYZ. Additionally, the ratio of the nominal to modified GENIE models is shown.}
  \label{fig:t2k_impact}
\end{figure}
Figure~\ref{fig:t2k_impact} shows the $\nu_{\mu}$--H$_{2}$O CC0$\pi$ event rate expected at the T2K near and far detectors for a fixed exposure, with and without modifications to the axial form factor. The nominal GENIEv3 10a\_02\_11a uses a dipole axial form factor with $M_{\mathrm{A}} = 0.941$ GeV$^2$ obtained through a fit to bubble chamber data~\addcite{}. The alternative model shown differs only in the use of the z-expansion model for the axial form factor, with parameters tuned to the LQCD results described in Sectoin~XYZ. The axial form factor change is relevant for all CCQE events, which clearly make up the vast majority of T2K's CC0$\pi$ samples, and it is apparent from Figure~\ref{fig:t2k_impact} that the total event rate has significantly changed as a result, by approximately 20\% in both cases.

It is also clear from the ratios that the change in the total event rate is not purely a normalization change, which is unsurprising as the change only affects CCQE events. However, the ratios are not the same for the near and far detectors, because the relative fractions of the different interaction modes that populate the CC0$\pi$ sample is not the same in the oscillated flux, and they have different $E_{\nu}^{\mathrm{true}}$ --- $E^{\mathrm{rec,\;QE}}_{\nu}$ relationships. This is problematic in a subtle way. In an oscillation analysis, the near detector data serves to constrain the values of all systematic uncertainties that affect the flux, cross section and detector systematic models. If a model is deficient, the systematics will be varied in a way that compensates for that deficiency as far as is possible. This is a valuable way to improve an {\it a priori} model with high-statistics data, but, it introduces the possibility for adding a bias if the deficiency is attributed to the wrong interaction mode. This is very likely to happen if different interaction modes have different relative freedom to change due to larger systematic uncertainties.



\begin{figure}[htbp]
  \centering
  \subfloat[ND]{\includegraphics[width=0.3\textwidth]{plots/DUNE_numu_Ar40_breakdown.pdf}}\hspace{75pt}
  \subfloat[FD]{\includegraphics[width=0.3\textwidth]{plots/DUNE_osc_numu_Ar40_breakdown.pdf}}
  \caption{DUNE...}
  \label{fig:dune_impact}
\end{figure}



% ------------------------------------------------------------------------------
% Future
\section{Future Improvements\label{sec:future}}
{\color{red}items to comment on}
\begin{itemize}
\item won't be performing precision calculations at 2-3 fm
\item variational with pi-N operators
\item this enables Breit-Frame and improved single-nucleon
\item domain-decomposed idea of Giusti et al
\end{itemize}


Precise nucleon form factors for (quasi)elastic scattering are of the most immediate concern
 for neutrino oscillation experiments and simultaneously one of the easiest
 targets for lattice QCD in regards to nucleon interactions.
However, processes with higher energy transfers are also a cause for concern,
 especially when considering the higher-energy neutrino flux at DUNE (see Figure~\ref{fig:dune_impact}).
These larger energy transfers can access other fundamentally different interaction topologies,
 such as resonant or nonresonant pion production mechanisms,
 nuclear responses with correlated nucleon pairs,
 or scattering off partons within nucleons.
In principle, all of these interaction mechanisms are accessible to lattice QCD,
 though with varying degrees of difficulty.
Given the discrepancy between lattice axial form factor data and experimental constraints,
 it is not unreasonable to expect other interaction mechanisms have similar discrepancies
 between theory and observation.
These interaction types are more challenging to extract or even inaccessible to
 experimental measurements.
Appeals to model assumptions may give some handle for missing quantities,
 but are also subject to unquantifiable systematic effects.

Calculations that access the combined resonant and nonresonant scattering amplitudes
 are the most similar to those of elastic scattering,
 where a current induces a transition of the nucleon to a multiparticle final state.
Only stable final states, such as nucleon-pion states or other multiparticle states,
 are obtained as eigenstates of the Hamiltonian.
Resonance properties are indirectly probed by converting the observed spectrum,
 with its power-law finite volume corrections, to a scattering phase shift in infinite volume.
These power-law finite volume corrections are in contrast with exponentially-damped
 corrections obtained for single-particle states.
The multiparticle states make up a dense spectrum that arises from
 states where individual hadrons move with different discrete lattice momenta.

The main difficulties in these calculations stem the challenges of extracting
 information about many excited states rather than a single ground state.
Empirical evidence from dedicated computations demonstrates that these multiparticle states
 are in practice difficult to quantify without interpolating operators constructed to
 closely resemble the states they are intended to identify,
 both in the number of quarks and antiquarks and also the individual
 momenta of the quark-level components~\textcolor{red}{[cite JLab]}.
The necessary interpolating operators for extracting multiparticle states often have
 unfavorable combinatoric factors multiplying the number of terms to account for
 permutations of quark lines, changing momenta, varying time ranges, or the like.
Some form of approximate all-to-all method is likely necessary
 to compute amplitudes of nucleon resonant and nonresonant transitions
 to states with pions.

The aforementioned issues are, at least in part, circumvented by applying approximate
 all-to-all techniques, for instance distillation~\textcolor{red}{[citations]}.
Though the overhead for these methods is large,
 these techniques enable sophisticated calculations with large bases
 of interpolating operators including multiparticle interpolators.
Using an all-to-all method specifically avoids the need to produce sequential
 quark propagator inversions that are used in traditional three-point correlator functions,
 which are the source of many unfavorable combinatoric factors and are
 costly when enumerating all possible quark line combinations.
Though the lowest resonances will be accessible with these techniques,
 the higher-mass resonances will be more technically difficult
 due to the increased density of states with the energy transfers involved.
Using these propagator data, a calculation of the Roper resonance contribution
 has the secondary consequence of offering a handle on removing excited
 state contamination from a calculation of the quasielastic form factors.
\textcolor{red}{[summarizing statement or two about current status]}

To access higher resonance states in the so-called shallow inelastic scattering (SIS) regime,
 lattice calculations will likely be more successful using other methods
 that access the inclusive nucleon scattering amplitude to hadronic states.
A four-point function calculation on the lattice can obtain a discrete sampling
 of the scattering amplitude as a function of the center-of-mass energy,
 convolved with weights that are exponential in the energy and time separation.
Applying an inverse transformation to obtain the scattering amplitude
 is an ill-posed problem, but initial calculations using techniques
 such as Backus-Gilbert or moments have shown promise.
\textcolor{red}{[more detail, references, check wording]}

Calculations of two-nucleon matrix elements provide key insights
 about the correlations between nucleons inside of a nuclear medium,
 a vital ingredient for construction of an effective theory
 of neutrino interactions with nuclear targets.
%% heavier than physical Mpi => NPLQCD/HALQCD controversy
First efforts have been made to compute deuteron and dineutron scattering phase shifts
 at unphysically heavy pion masses.
There is some controversy about whether the deuteron forms a
 bound state at these higher pion masses.~\textcolor{red}{[citations, more content]}.

%% gA(Q2) on D2 and nuclear ET
Future calculations of matrix elements for currents inserted between
 two-nucleon states could provide direct information about the LEC inputs to nuclear
 models~\cite{Drischler:2019xuo}
 or by providing direct comparisons of neutrino interaction matrix elements in deuterium.
\textcolor{red}{[more detail needed on ET LECs]}
%% overlap of energy scales of NN vs NNpi
\textcolor{red}{[the following is likely too much detail]}
Deuterium corrections from nuclear models were assumed to be strong only at low momentum transfer
 and energy-independent in the reanalysis of deuterium bubble chamber data~\cite{Meyer:2016oeg},
 despite the inability of these corrections to account for the theory-data discrepancies.
A direct lattice QCD computation of these effects would isolate the effect,
 either by definitively attributing the discrepancy to deuterium effects
 or by implicating the other systematics corrections.



% ------------------------------------------------------------------------------
% Conclusions
\section{Conclusions\label{sec:conclusions}}
LQCD collaborations are able to produce consistent results for benchmark quantities such as $g_{\mathrm{A}}$ with percent level systematic uncertainties, which are in excellent agreement with experimental data.
These results introduce the exciting possibility of LQCD calculations to tackle other quantities for which are not easily experimentally accessible, or for which tensions between measurements, or competing models, exist.
In this review we discussed using LQCD to calculate nucleon form factors as a function of momentum transfer, which are of particular interest to the few-GeV neutrino experimental program.
Important tensions exist in current parameterizations of the vector form factors, but here we focused on the axial form factor, $F_{\mathrm{A}}(Q^2)$, which is of primary importance because current parameterizations of it are simplistic and rely on a handful of low-statistics $\nu N$ bubble chamber measurements.
It cannot be cleanly measured with existing experiments which use heavier nuclear targets for safety reasons and to increase the event rate, so LQCD offers a novel path to this important quantity.
We have compared $F_{\mathrm{A}}(Q^2)$ calculations from a variety of different LQCD collaborations using different approaches and techniques, and shown them to be in good agreement with each other, but crucially, in poor agreement with the simple dipole model tuned to historic $\nu N$ data currently relied upon.
Assuming that no systematic effects affecting all of the LQCD calculations are uncovered, this suggests a significant increase of approximately 20\% to the strength of the CCQE scattering channel that dominates the neutrino scattering cross section for $E_{\nu} \lesssim 1$ GeV.
We have demonstrated that these results produce a significant change in the predicted neutrino event spectra for T2K (which has similar considerations to Hyper-K) and DUNE experiments.
Determining the impact on oscillation results would require a full analysis performed by each experimental collaboration, but it is clear that LQCD results for $F_{\mathrm{A}}(Q^2)$ may offer a valuable insight that can clarify aspects of the complex neutrino interaction modeling problem these experiments face.
We additionally discussed a number of ways in which current calculations can be improved and validated, to increase confidence that the LQCD results are not subject to an uncontrolled systematic uncertainty, and to reduce systematics.
Finally, we discussed a number of other quantities which are important to neutrino oscillation experiments in the few-GeV energy regime which LQCD can provide data for, for which there are no experimental solutions, including resonant pion production at higher energy transfers, which is of particular interest to the DUNE experimental program, and insights into nucleon-nucleon correlations.
These possibilities would all work to break degeneracies and overcome experimental challenges that affect current $\nu A$ interaction modelling efforts.

%% \cw{Added some very rough bullet points below}
%% \begin{itemize}
%% \item LQCD collaborations are now producing consistent results for a number of benchmark quantities, which are in agreement with experimental data.
%% \item Able to calculate nucleon form factors with current techniques. Results for FA from different collaborations using different approaches are in agreement --- unlikely to be a common missing systematic uncertainty
%% \item LQCD FA results differ from experimental measurements from old nu-H and nu-D bubble chambers. These are important quantities for current neutrino oscillation experiments, but cannot be measured cleanly as all modern neutrino scattering and oscillation experiments use heavy nuclear targets for safety reasons and to increase the event rate.
%% \item We have demonstrated how these results produce a significant change for T2K (Hyper-K) and DUNE experiments. Determining the impact on oscillation results would require a full oscillation analysis performed by each experimental collaboration, but clear that LQCD can reduce uncertainties.
%% \item There are a number of other quantities which are important to neutrino oscillation experiments in the few-GeV energy regime which LQCD can provide data for, for which there are no experimental solutions.
%% \end{itemize}



%Disclosure
\section*{DISCLOSURE STATEMENT}
While the authors collaboration affiliations do not affect the objectivity of this review, we wish to include them for transparency. ASM and AWL are both current members of the CalLat collaboration. ASM is a current member of the Fermilab Lattice and MILC collaborations. CW is a current member of the T2K and DUNE collaborations.
The authors are not aware of any funding, or financial holdings that might be perceived as affecting the objectivity of this review.

% Acknowledgements
\section*{ACKNOWLEDGMENTS}
The work of ASM was supported by the Department of Energy, Office of Nuclear Physics, under Contract No. DE-SC00046548.
The work of AWL and CW was supported by the Director, Office of Science, Office of Basic Energy Sciences, of the U.S. Department of Energy under Contract No. DE-AC02-05CH11231.

%% Supposedly, we should always include these things as LBNL employees: https://commons.lbl.gov/display/rpm2/Scientific+and+Technical+Publications+Requirements#myId--1898802862 (section D.2)
%This document was prepared as an account of work sponsored by the United States Government. While this document is believed to contain correct information, neither the United States Government nor any agency thereof, nor the Regents of the University of California, nor any of their employees, makes any warranty, express or implied, or assumes any legal responsibility for the accuracy, completeness, or usefulness of any information, apparatus, product, or process disclosed, or represents that its use would not infringe privately owned rights. Reference herein to any specific commercial product, process, or service by its trade name, trademark, manufacturer, or otherwise, does not necessarily constitute or imply its endorsement, recommendation, or favoring by the United States Government or any agency thereof, or the Regents of the University of California. The views and opinions of authors expressed herein do not necessarily state or reflect those of the United States Government or any agency thereof or the Regents of the University of California.

%This manuscript has been authored by an author at Lawrence Berkeley National Laboratory under Contract No. DE-AC02-05CH11231 with the U.S. Department of Energy. The U.S. Government retains, and the publisher, by accepting the article for publication, acknowledges, that the U.S. Government retains a non-exclusive, paid-up, irrevocable, world-wide license to publish or reproduce the published form of this manuscript, or allow others to do so, for U.S. Government purposes.

% References
\bibliography{AR_review.bib}
\bibliographystyle{ar-style5}
%ArXiv references may be formatted as follows, in the Literature Cited section: “1. Author A, Author B. arXiv:XXXX.XXXX [hep-ph] (2017)”

\end{document}
